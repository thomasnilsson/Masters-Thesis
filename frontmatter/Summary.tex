%!TEX root = ../Thesis.tex
\chapter{Summary}
Existing contributions within the field of generating mobility features and context are cumbersome, if even possible, to reproduce with regards to their source code. Furthermore, the algorithms that exist in the literature do not necessarily lend themselves to the real-time computation of features. These two research problems were addressed in this thesis. The solution was a software library implemented in the Flutter framework. Given that the programmer collects location data themselves, mobility can be generated using the package with just 3 lines of Dart code. In addition the package was tested in a field study, where a mobile application running the package was used by 10 participants. The app collected the participants location several thousand times per day and participants would fill out a small questionnaire each day pertaining to the features. In comparing the answers to the computed features, promising results were achieved for the participants which were diligent in tracking their location and providing answers. The results were however not consistent for those participants with low participation in terms of data collected and answers provided.