%!TEX root = ../Thesis.tex
\chapter{Summary}
Research has shown that mobility features derived from location data can be used to describe behavior changes in people suffering from depression-related diseases. Existing contributions dealing with the computation of mobility features are cumbersome to reproduce due to a lack of publicly available source code. Furthermore, the algorithms provided have not been considered for real-time computation on a smart-phone. This means they cannot necessarily be used on demand to support psychotherapeutic interventions. These two research problems were addressed in this thesis, by implementing the Mobility Features Package - a software package for the Flutter framework. The package supports the computation of the features \textit{Home Stay, Number of Places, Distance Travelled, Location Variance, Entropy, Normalized Entropy,} and \textit{Routine Index}. The package makes it possible for an application programmer to compute these features with just 3 lines of code. Furthermore, the application programmer can flexibly choose any Flutter plugin for tracking location data. This makes it very easy for researchers within mobile health to include mobility features in their mobile health applications.\\

A field study was conducted with 10 participants in which their location was tracked via an iOS application over 3 weeks. This application used the Mobility Features Package to compute features several times a day. Participants filled out a daily questionnaire pertaining to 3 of the features, and these answers were afterward compared to features computed during the study. We found that the Mobility Features Package computes the Number of Places visited with an RMSE of 0.99 places, the Home Stay percentage with an RMSE of 14.3\%, and the Routine Index with an RMSE of 22.5\%. However, many non-uniform gaps were observed in the collected location data which impacted the RMSE of the features. For future improvement of the package, it is highly relevant for the algorithms to use an imputation method to handle missing data.
