%!TEX root = ../Thesis.tex
\chapter{Summary}
Mobility features derived from location data can be used to describe behaviour changes in people suffering from depression-related diseases. Existing contributions within the field of computing mobility features in smart-phone environment are cumbersome, if even possible, to reproduce with regards to their source code. Furthermore, the algorithms for computing these features in the literature do not always lend themselves to be computed in real-time. These two research problems were addressed in this thesis for which the solution was a software library implemented in the Flutter framework. The final version of the package enables the programmer to generate a set of mobility features with just 3 lines of code. A field study was conducted where a mobile application running the package was used by 10 participants. Participants filled out a daily questionnaire pertaining to the features, these answers were then compared to the computed features. In comparing the answers to the computed features, promising results were achieved for the participants which were diligent in tracking their location and providing answers. 