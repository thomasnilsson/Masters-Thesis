%!TEX root = ../Thesis.tex
\chapter{Summary}
Mobility features derived from location data can be used to describe behaviour changes in people suffering from depression-related diseases. Existing contributions within the field of computing mobility features in smart-phone environment are cumbersome, if even possible, to reproduce with regards to their source code. Furthermore, the algorithms for computing these features in the literature do not always lend themselves to be computed in real-time. These two research problems were addressed in this thesis for which the solution was a software package implemented in the Flutter framework. Through the package, an application programmer is able to compute mobility features with just 3 lines of code. Furthermore the package allows the application programmer to flexibly choose their own Flutter plugin for tracking location data to avoid dependency issues with other plugins. A field study was conducted where a mobile application running the package was used by 10 participants over 3 weeks. Participants filled out a daily questionnaire pertaining to the features, these answers were then compared to the computed features. When comparing the daily location features with subjective user data we found that the Mobility Features Package predicts the Number of Places visited with an RMSE of 0.5 places, the Home Stay percentage with an RMSE of 14.3\% and the Routine Index with an RMSE of 22.5\%. However, many non-uniform gaps were observed in the collected location data which impacted the results of the study. For future improvement it will be highly relevant to use an imputation method to mitigate the issue of missing data.
