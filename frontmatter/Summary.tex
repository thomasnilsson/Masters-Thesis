%!TEX root = ../Thesis.tex
\chapter{Summary}
Mobility features derived from location data can be used to describe behaviour changes in people suffering from depression-related diseases. Existing contributions within the field of generating mobility features are cumbersome
to reproduce due to a lack of publicly available source code. Furthermore, the
algorithms that exist in the literature have not been considered for real-time computation which has implications on algorithm design and resource constraints. These two research problems were addressed in this thesis by the Mobility Features Package, a software package for the Flutter framework. Through the package, an application programmer is able to compute mobility features with just 3 lines of code, making it very easy for mHealth researchers to compute mobility features in their studies. Furthermore the package allows the application programmer to flexibly choose their own Flutter plugin for tracking location data to avoid dependency issues with other plugins.\\ 

A field study was conducted in which a mobile application using the package ran on the phones of 10 participants over 3 weeks. Participants filled out a daily questionnaire pertaining to 3 of the features and these answers were afterwards compared to features computed during the study. In comparing the daily location features with subjective user data we found that the Mobility Features Package predicts the Number of Places visited with an RMSE of 0.5 places, the Home Stay percentage with an RMSE of 14.3\% and the Routine Index with an RMSE of 22.5\%. However, many non-uniform gaps were observed in the collected location data which impacted the accuracy of the features. For future improvement it will be highly relevant for the package to use an imputation method to handle missing data.
