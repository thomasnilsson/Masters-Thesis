%!TEX root = ../Thesis.tex
\chapter{Summary}
Existing contributions within the field of generating mobility features and context are cumbersome, if even possible, to reproduce with regards to their source code. Furthermore, the algorithms that exist in the literature do not necessarily lend themselves to the real-time computation of features. These two research problems were addressed in this thesis for which the solution was a software library implemented in the Flutter framework. The final version of the package enables the programmer to generate a set of mobility features with just 3 lines of code. A field study was conducted where a mobile application running the package was used by 10 participants. The application collected the participants' location several thousand times per day and participants would fill out a small questionnaire each day pertaining to the features. In comparing the answers to the computed features, promising results were achieved for the participants which were diligent in tracking their location and providing answers. 