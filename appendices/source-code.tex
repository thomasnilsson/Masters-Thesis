\chapter{Source Code}
\label{appendix:source-code}
\subsection{Storing and Loading Data}
The storing and loading of data, which includes Location Samples, Stops, and Moves happen through the \textit{MobilitySerializer} class. This class allows classes that implement the Serializable interface to be serialized and de-serialized. Just like the GeoSpatial interface, the Serializable interface is also implemented as a private abstract class only used internally in the package library. The interface contains a method for serializing a class to JSON, named \verb|toJson()| which takes no parameters and produces a HashMap of Strings to the dynamic, the dynamic type meaning any type. This is the Dart equivalent of a JSON object. Another method the interface forces other classes to implement is the deserialization method \verb|fromJson(json)| which takes a JSON object as a parameter and creates a runtime object of the given type, from the JSON object. The implementation of this method is left to the individual classes implementing the interface which is done by extracting data from the JSON object.

\begin{minted}{dart}
    abstract class _Serializable {
      Map<String, dynamic> _toJson();
    
      _Serializable._fromJson(Map<String, dynamic> json);
    }
\end{minted}

The MobilitySerializer class is a generic which allows the type \verb|E| to be specified later, with \verb|E| referring to either a Location Sample, Stop or Move which all implement the Serializable interface. The MobilitySerializer is constructed using a reference to a File object. The File object is used for storing the data of the given type i.e. Location Samples are stored one file, Stops in another, and Moves in a third.

\begin{minted}{dart}
    class MobilitySerializer<E> {
      File file;
      
      MobilitySerializer._(this.file) {
        bool exists = file.existsSync();
        if (!exists) {
          flush();
        }
      }
      
      Future<void> flush() async =>
          await file.writeAsString('', mode: FileMode.write);
    ...
    }
\end{minted}

When initialized, it is checked whether or not the specified file exists, and if not the \verb|flush| method is called, which simply writes an empty string to the file, overriding any content, which has the effect of creating the file, should it not already exist. A concrete example of instantiated the MobilitySerializer for Stops is shown below, where \textit{stops.json} refers to the file in which Stops should be stored.

\begin{minted}{dart}
    MobilitySerializer<Stop> stopSerializer =
            MobilitySerializer<Stop>._(await _file('stops.json'));
\end{minted}

For storing data the \verb|save| method is used which takes in a list of objects which all implement the Serializable interface. Each element in the list is serialized via its \verb|toJson| method and concatenated into one big string separated by a delimiter token, and this string is then written to the specific file of the MobilitySerializer object.

\begin{minted}{dart}
    Future<void> save(List<_Serializable> elements) async {
      String jsonString = "";
      for (_Serializable e in elements) {
        jsonString += json.encode(e._toJson()) + delimiter;
      }
      await file.writeAsString(jsonString, 
        mode: FileMode.writeOnlyAppend);
    }
\end{minted}

Loading works in the reverse order, where the contents of the specified file are loaded into a string, the string is then split into elements using the delimiter token and each of these elements is de-serialized using the \verb|fromJson| method.  For deciding which type to de-serialize the elements into, a switch statement is used that checks the type of \verb|E| which is specified when the MobilitySerializer object is instantiated.

\begin{minted}{dart}
    Future<List<_Serializable>> load() async {
        String content = await file.readAsString();
    
        List<String> lines = content.split(delimiter);
    
        Iterable<Map<String, dynamic>> jsonObjs = lines
            .sublist(0, lines.length - 1)
            .map((e) => json.decode(e))
            .map((e) => Map<String, dynamic>.from(e));
    
        switch (E) {
          case Move:
            return jsonObjs.map((x) => 
                Move._fromJson(x)).toList();
          
          case Stop:
            return jsonObjs.map((x) => 
                Stop._fromJson(x)).toList();
          
          default:
            return jsonObjs.map((x) => 
                LocationSample._fromJson(x)).toList();
        }
    }
\end{minted}

Ideally, the switch statement could have been replaced by the following one-liner:
\begin{minted}{dart}
    return jsonObjs.map((x) => E.fromJson(x)).toList();
\end{minted}

However, this relies on the language feature called reflection \footnote{\url{https://www.javaworld.com/article/2075801/reflection-vs--code-generation.html}} which allows the compiler to infer the type of \verb|E| at compile-time. However, Dart does not support \textit{reflection} which makes this impossible.


\subsubsection{Accessing the File System}
For storing collected location data the MobilitySerializer for Location Samples can be retrieved through this class, with a getter method. 
);
\begin{minted}{dart}
static Future<MobilitySerializer<LocationSample>>
      get locationSampleSerializer async =>
          MobilitySerializer<LocationSample>._(await _file(LOCATION_SAMPLES)
\end{minted}

Internally this class has a method for creating a file system reference, which relies on the platform the application is running on. For mobile apps, the file system must be accessed through the \verb|path_provider| package with the \verb|getApplicationDocumentsDirectory()| method. If the application is run on the desktop, such as when unit testing the file system can be accessed by specifying a file name directly. This is a textbook example of hiding complexity from the application programmer.

\begin{figure}
    \centering
    \begin{minted}{dart}
    static Future<File> _file(String type) async {
      bool isMobile = Platform.isAndroid || Platform.isIOS;
    
      String path;
      if (isMobile) {
        path = (await getApplicationDocumentsDirectory()).path;
      } else {
        path = 'test/data';
      }
      return new File('$path/$type.json');
    }
    \end{minted}
    \caption{Lazy evaluation of a feature}
    \label{fig:lazy-evaluation}
\end{figure}

