%!TEX root = ../Thesis.tex
\chapter{Questionnaires}
\label{appendix:questionnaires}
\subsection{PHQ-9 (Patient Health Questionnaire)}
Patented by Pfizer

The PHQ-9 questionnaire contains 9 questions pertaining to the mental state of the patient \footnote{\url{https://patient.info/doctor/patient-health-questionnaire-phq-9}}. Each question asks ‘Over the last two weeks, how often have you been bothered by any of the following problems?’ with the questions being the following: \\

\textbf{Q1:} Little interest or pleasure in doing things? \\

\textbf{Q2:} Feeling down, depressed, or hopeless? \\

\textbf{Q3:} Trouble falling or staying asleep, or sleeping too much? \\

\textbf{Q4:} Feeling tired or having little energy? \\

\textbf{Q5:} Poor appetite or overeating? \\

\textbf{Q6:} Feeling bad about yourself, or that you are a failure, or have let yourself or your family down? \\

\textbf{Q7:} Trouble concentrating on things, such as reading the newspaper or watching television? \\

\textbf{Q8:} Moving or speaking so slowly that other people could have noticed. Or the opposite – being so fidgety or restless that you have been moving around a lot more than usual? \\

\textbf{Q9:} Thoughts that you would be better off dead, or of hurting yourself in some way? \\

Each question can be answered with the following 4 possibilities, each giving a number of points indicated in brackets:

\begin{itemize}
\item Not at all (0 points)
\item Several days (1 point)
\item More than half the days (2 points)
\item Nearly every day (3 points)
\end{itemize}

At the end of the survey, the points are summed up and the patient is categorized into one of 5 categories based on the number of points acquired:

\begin{itemize}
\item Less than 5 (no depression)
\item 5-9 (mild depression)
\item 10-14 (moderate depression)
\item 15-10 (moderate/severe depression)
\item Greater than 20 (severe depression)
\end{itemize}


\subsection{EMA (Ecological Momentary Assessment)}
The EMA Questionnaire contains six short questions each targeting a common mental state strongly related to depression, each for which the user has to give a self-perceiving rating from 1 (none) to 7 (extreme):

\begin{itemize}
\item Negative Affect
\item Hopelessness
\item Anhedonia (loss of interest)
\item Fatigue/Energy
\item Loneliness
\item Positive Affect
\end{itemize}