\chapter{Conclusion}
\label{chapter:08}

For the conclusion we shall address the original three research questions which made up the hypothesis:

\subsubsection*{Which mobility features are relevant to include in a software package?}
The features Number of Places, Home Stay, Entropy/Normalized Entropy, Location Variance, and Routine Index were chosen based on the work by Saeb. et al \cite{Saeb2015} and Canzian et. al \cite{Canzian2015}. The most important features pertaining to depression were \textit{Home Stay}, \textit{Entropy/Normalized Entropy}, and the \textit{Routine Index}. All features except for the \textit{Routine Index} can be evaluated daily without the need for historical data. In addition, a set of intermediate features, namely \textit{Stops}, \textit{Places}, and \textit{Moves} were also implemented as part of the package. These intermediate features aided in reducing the amount of data processed by the feature-extraction algorithms and proved useful as features themselves.

\subsubsection*{How can these features be computed in real-time, on a smartphone device?}
A combination of mathematical definitions and clever data storage and loading was necessary to achieve real-time feature computation: All features are computed using Location Samples collected from the current day. Certain mathematical re-definitions had to be made for the features such that they could be computed given a dataset from an incomplete day. A novel definition for the \textit{Routine Index} feature was in which \textit{Stops} from multiple days are used to compute the feature. These \textit{Stops} must be saved on the device and loaded whenever the feature computation takes place. \textit{Stops} effectively work as a compressed form of Location Samples and thus reduces the storage and allowed feature computation to be possible in real-time.

\subsubsection*{How does the design of such a software package look like?}
It was decided upon a design that provides an API with a high abstraction level that hides most of the feature computation implementation from the user. This design allows the programmer to compute the features with just 3 lines of code, excluding data collection. The implementation details hidden away from the user included storing and loading historical data as well as computing features using the historical- and daily data. 

\subsubsection*{Validation of the Package}
In addition to the three main questions, the package was validated using unit testing which tested the accuracy of the algorithms for simple synthetic data sets. For validating the algorithms and the package in a broader sense, a field study with 10 participants was conducted. For this study, an application was developed in Flutter which used the package. The demo application was distributed digitally via Apple's \textit{Testflight} app which allowed for quick patching of bugs with minimal installation trouble to the users. It was the development of the study app which lead to the final package design. The application used a version of the package which had a lower level of abstraction which made it much more cumbersome to use which made it apparent that the abstraction level needed to be much higher. The study produced a dataset of 240 MB of Location Samples spread over 10 participants with subjective answers. It would have been possible, if time allowed, to perform parameter tuning for the feature algorithms to improve their accuracy. This would likely require manual labeling of the data since the subjective data is somewhat unreliable which would be a considerable time sink.