\chapter{Conclusion}
\label{chapter:08}

For the conclusion we shall address the original three research questions which made up the hypothesis:

\subsubsection*{Which mobility features are relevant to include in a software package?}
The features \textit{Number of Places, Home Stay, Entropy/Normalized Entropy, Location Variance}, and the \textit{Routine Index} were chosen based on the work by Saeb. et al \cite{Saeb2015} and Canzian et. al \cite{Canzian2015}. An additional set of features, namely Stops, Places and Moves were included as well based on the work by Cuttone et al. \cite{sparse-location-2014} and Canzian et al. 

\subsubsection*{How can these features be computed in real-time, on a smartphone device?}
The features needed to be re-defined such that they could be computed from an incomplete dataset. All features except for the Routine Index can be without the need for historical data. A novel definition for the Routine Index feature was made that required Stops from multiple days, stored on the device, to compute the feature. These Stops must be saved on the device when computed, and are loaded again whenever the Routine Index is calculated in the future.

\subsubsection*{How does the design of such a software package look like?}
It was decided upon a design that provides an API with a high abstraction level that hides most of the feature computation implementation from the user. This design allows the programmer to compute the features with just 3 lines of code, excluding data collection. The implementation details hidden away from the user included storing and loading historical data as well as computing features using the historical- and daily data. The package does not depend on any specific location plugin due to its design, which allows the programmer to flexibly choose their own plugin for tracking location data. Being independent of any specific location plugin enables easier maintenance and allows the package to be used among other packages dependent on location tracking, without causing dependency issues. 

\subsubsection*{Validation of the Package}
Through a field study with 10 participants, the capabilities of the package were demonstrated. For this study a Flutter application collected the participants' location for 3 weeks and used the package to compute the participants' features multiple times daily. Participants also filled out a daily questionnaire pertaining to the features, which were compared to the computed features. In the 3-week study, the following insights concerning the Mobility Features Package were drawn:

\begin{itemize}
    \item The Mobility Features Packages successfully allowed mobility features to be computed several times a day. However, non-uniform gaps were observed in the collected location data, reducing the accuracy of the computed features. 
    
    \item When comparing the daily features with subjective user data we found that the Mobility Features Package computes the Number of Places visited with an RMSE of 0.5 places, the Home Stay percentage with an RMSE of 14.3\% and the Routine Index with an RMSE of 22.5\%.
    
    \item The algorithms tends to undershoot in computing the Number of Places and Home Stay, which is likely due to gaps in the collected data. There is no consistent over- or undershooting shown when it comes to computing the Routine Index. This likely stems from the between-subject variance in the routine answers as well as the gaps in the location data.
\end{itemize}

Different approaches to mitigate the errors in computing features, i.e. how the feature algorithms can be improved as well as the method which saves and loads historical data. In addition it is also discussed how an imputation method using historical data can be used to cover the observed gaps in the location data.