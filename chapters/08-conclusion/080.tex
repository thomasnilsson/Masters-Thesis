\chapter{Conclusion}
\label{chapter:08}

For the conclusion we shall address the original three research questions which made up the hypothesis:

\subsubsection*{Which mobility features are relevant to include in a software package?}
The features Number of Places, Home Stay, Entropy/Normalized Entropy, Location Variance, and Routine Index were chosen based on the work by Saeb. et al \cite{Saeb2015} and Canzian et. al \cite{Canzian2015}. The most important features pertaining to depression were \textit{Home Stay}, \textit{Entropy/Normalized Entropy}, and the \textit{Routine Index}. All features except for the \textit{Routine Index} can be evaluated daily without the need for historical data. In addition, a set of intermediate features, namely \textit{Stops}, \textit{Places}, and \textit{Moves} were also implemented as part of the package. These intermediate features aided in reducing the amount of data processed by the feature-extraction algorithms and proved useful as features themselves.

\subsubsection*{How can these features be computed in real-time, on a smartphone device?}
A combination of mathematical definitions and clever data storage and loading was necessary to achieve real-time feature computation: All features are computed using Location Samples collected from the current day. Certain mathematical re-definitions had to be made for the features such that they could be computed given a dataset from an incomplete day. A novel definition for the \textit{Routine Index} feature was made which required \textit{Stops} from multiple days, stored on the device, to compute the feature. These \textit{Stops} must be saved on the device and loaded whenever the feature computation takes place. \textit{Stops} effectively work as a compressed form of Location Samples and thus reduces the storage and allowed feature computation to be possible in real-time.

\subsubsection*{How does the design of such a software package look like?}
It was decided upon a design that provides an API with a high abstraction level that hides most of the feature computation implementation from the user. This design allows the programmer to compute the features with just 3 lines of code, excluding data collection. The implementation details hidden away from the user included storing and loading historical data as well as computing features using the historical- and daily data. The package does not depend on any specific location plugin due to its design, which allows the programmer to flexibly choose their own plugin for tracking location data. Being independent of any specific location plugin enables easier maintenance and allows the package to be used among other packages dependent on location tracking, without causing dependency issues. 

\subsubsection*{Validation of the Package}
Through a field study with 10 participants, the capabilities of the package were demonstrated. For this study a Flutter application collected the participants' location for 3 weeks and used the package to compute the participants' features multiple times daily. Participants also filled out a daily questionnaire pertaining to the features, which were compared to the computed features. In the 3-week study, the following insights concerning the Mobility Features Package were drawn:

\begin{itemize}
    \item The Mobility Features Packages successfully allowed mobility features to be computed several times a day.
    
    \item Non-uniform gaps were observed in the collected location data (as in similar 'in the wild' studies such as in \cite{detecting_bipolar_from_gps}) which reduced the accuracy of the computed features. These gaps are likely due to the operating system on the smart-phones throttling background tracking.
    
    \item When comparing the daily location features with subjective user data we found that the Mobility Features Package predicts the Number of Places visited with an RMSE of 0.5 places, the Home Stay percentage with an RMSE of 14.3\% and the Routine Index with an RMSE of 22.5\%.
    
    \item The algorithms tends to undershoot in predicting the number of places and home stay, likely due to gaps in the data. There is no consistent over- or undershooting done when it comes to computing the Routine Index, which likely stemmed from gaps in the location data, as well as variance the validity of the subjective answers regarding the participants routine.
\end{itemize}

Different approaches to mitigate the errors in computing features, i.e. how the feature algorithms can be improved as well as the procedure which saves and loads historical data. In addition it is also discussed how an imputation method using historical data is likely need to cover gaps in the location data which were observed in the data analysis.