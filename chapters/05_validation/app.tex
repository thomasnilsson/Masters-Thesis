\section{App Implementation}
Implemented in Flutter, gathered data from the user and computed features by using the Mobility Features package. 

\subsection{Diary}
Daily diary, four questions, diary could be opened from the main screen or via a notification. How many places today, home many hours away from home, visited any new places and how much today looked like the previous days.


\subsection{Location Tracking}
Location was tracked as often as possible, which was approx. once per second. Since it was developed in Flutter it is possible to produce an iOS- as well as an Android application. However, there were problems with the Android platform due to changes to the Location API made by google in early 2020, which limits the background location data functionality to only work if the application was kept in foreground. This requires the developer to use a foreground service to keep the application in foreground, which allegedly uses a lot of battery. It was tried with the package BackgroundLocator, however this package had compilation errors on the iOS platform. It was therefore decided to just use the application for iOS in lieu of the time limitation of the project.

\subsection{Device Storage}
JSON files stored on the device itself. Used to collect data before uploading

\subsection{Online Storage}
Firebase Storage, JSON files, participant ID

\subsection{Cloud Messaging}
Firebase Cloud Messaging (FCM) was used in order to remind users to fill out the diary daily. Users were sent a notification each day at 8 PM that when clicked would open the diary screen. 

\subsection{Distribution}
Apple TestFlight