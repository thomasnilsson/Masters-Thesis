% \section{Feature Discussion}


% % \subsection{Calculating Home Stay}
% % For calculating the Home Stay two-time quantities are used: The \textit{total time} spent at home and the\textit{total time spent at all places}, where the total duration spent at home can be derived from the Stops which were made at the home cluster. For computing the total amount of time spent, two approaches can be used: The simplest approach is using the time elapsed since midnight at the time of calculating the features. The other slightly more precise approach is to use the departure time of the last known Stop. The first approach will not take into account that the user, at the time of feature computation is likely still gathering data, which is not yet compressed into a stop. Since the data is not compressed into a stop yet, the data will therefore not count however the time which it took to gather this data will be used to calculate the total time. The more precise approach uses the timestamp of the departure of the last known stop, which is a tighter estimate. It can be done either way and will likely produce similar results unless an extreme edge case is considered where a Stop has not been found for a very long time due to the user transporting themselves. Here, the results would differ quite a lot but would also tell two different stories where one might be more correct than the other, depending on the meaning to be derived from the feature.

% \subsection{Defining the Routine Index}
% We define a routine as the repeating of a pattern - in this case in terms of the places visited at a certain time of the day. This includes where how much time is spent at home and when. However, most people will go on vacation during the year, which means the place where they sleep changes. In general, peoples' habits will inevitably change somewhat over time, and if one compares the routine of a certain person now to what their routine looked like a year ago, it is not unlikely to be very different. However just because someone changes their routine over time, does not mean they don't currently possess one. Therefore it was chosen to base the \textit{Routine Index} on the last 4 weeks (28 days) of data in order to base the routine overlap on more recent days. An issue that was not considered for this thesis is the fact that the routine on weekdays differs a lot from the routine during the weekend. This is especially true for people who spent 8 or more hours at work during the weekdays and spent those 8 hours somewhere else during Saturday and Sunday since it means the \textit{Routine Index} cannot exceed $\frac{2}{3}$ due to a third of the day's total hours being spent at a different place than usual. To add to this, even weekdays may look slightly different from one another, especially for those who are part of sports clubs which meet during certain days of the week. In future work, it would be interesting to investigate whether or not comparing Mondays to Mondays and vice versa for every day in the week would yield more accurate results. In addition, the \texit{Routine Index} cannot rely on a full day of data if it is to be calculated in real-time. To make the feature represent something meaningful given an incomplete day of data, it should reflect the routine of the user up until the current time of the day. This means if it is calculated at 14:00 then it should only take into consideration the data from the first 14 hours from previous days as well. Because of this, the \textit{Routine Index} may be high early in the day since people usually sleep the same place, but are open to deviating as the day progresses. This variance in real-time can be useful to an application programmer in a recommender-system setting, where a trigger based on the \textit{Routine Index} can be set, such that the user is alerted when the value falls below a certain threshold.

