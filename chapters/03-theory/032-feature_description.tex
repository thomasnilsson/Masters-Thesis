\section{Feature Descriptions}
The Mobility Features which were used were a subset of the features discussed by \cite{Saeb2015,Canzian2015}. In addition a set of, what we will refer to as \textit{intermediate features} were also used which from the work of \cite{sparse-location-2014}. 

\subsection{Intermediate Features}
Common for many of the algorithms for finding user mobility features is that they rely on clustering of data points, in order to find the number of places. However when dealing with large amounts of data points it may be necessary to reduce the initial amount of data points such that these clustering algorithms are able to run faster. This down-sampling process will be carried out by clustering raw data points into what we shall refer to as \textit{Stops} indicating locations where the participant did not move around a lot. The \textit{Stop} notion is loosely based on the \cite{sparse-location-2014}. The pre-processing produces \textit{intermediate features} from which the final mobility features are derived. We define these \textit{intermediate features} as follows:

\subsubsection*{Stop}
A collection of GPS points which together represent a visit at a known \texit{Place} (see below) for an extended period of time. A \textit{Stop} is defined by a location that represents the centroid of a collection of data points, from which a \textit{Stop} is created. In addition a \textit{Stop} also has an \textit{arrival}- and a \textit{departure} time-stamp, representing when the user arrived at the place and when the user left the place. From the arrival- and departure timestamps of the \textit{Stop} the duration can be computed.

\subsubsection*{Place}
A group of stops that were clustered by the DBSCAN algorithm \cite{density-based-1996}. From the cluster of stops, the centroid of the stops can be found, i.e. the center location. In addition, it can be computed how long a user has visited a given place by summing over the duration of all the stops at that place.

\subsubsection*{Move}
The travel between two Stops, which the user will pass though a path of GPS points. The distance of a Move can be computed as the sum of using the haversine distance of this path. Given the distance travelled as well as departure and arrival timestamp from the Stops, the average speed at which the user traveled can be derived. 

\subsubsection*{Hour Matrix}
A matrix with 24 rows, each row representing an hour in a day, and columns equal to the number of places. The \textit{Hour Matrix} represents the time-place distribution for the user during a day.

\begin{figure}
    \centering
    \begin{tabular}{|l|l|l|l|l|}
    \hline
    \textbf{}        & \textbf{Place \#1} & \textbf{Place \#2} & \textbf{...} & \textbf{Place \#N} \\ \hline
    \textbf{00 - 01} &                    &                    &              &                    \\ \hline
    \textbf{01 - 02} &                    &                    &              &                    \\ \hline
    \textbf{...}     &                    &                    &              &                    \\ \hline
    \textbf{16 - 17} &                    &                    &              &                    \\ \hline
    \textbf{17 - 18} &                    &                    &              &                    \\ \hline
    \textbf{18 - 19} &                    &                    &              &                    \\ \hline
    \textbf{...}     &                    &                    &              &                    \\ \hline
    \textbf{23 - 00} &                    &                    &              &                    \\ \hline
    \end{tabular}
    \caption{An \textit{Hour Matrix}}
    \label{fig:time-table}
\end{figure}

\subsection{Feature Descriptions}
The features derived from this class are \textit{Home Stay (\%)}, \textit{Location Variance}, \textit{Number of Places}, \textit{Entropy} \textit{Normalized Entropy} , \textit{Distance Travelled} and the \textit{Routine Index (\%)}. These features are computed daily which for example means Home Stay refers to the portion of the time today which was spent at home. Below a short definition of each feature will be given since some of them differ somewhat from the descriptions given by \cite{Canzian2015,Saeb2015}. A more detailed description will be given in Section \ref{section:definitions}.

\subsubsection*{Home Stay}
The portion (percentage) of the total time elapsed since midnight which was spent at home. 

\subsubsection*{Location Variance}
The statistical variance in the latitude- and longitudinal coordinates.

\subsubsection*{Number of Places}
The number of places visited today.

\subsubsection*{Entropy}
The entropy with respect to time spent at places.

\subsubsection*{Normalized Entropy}
The normalized entropy with respect to time spent at places.

\subsubsection*{Distance Travelled}
The total distance travelled today (in meters), i.e. not limited to walking or running.

\subsubsection*{Routine Index}
The percentage of today that overlapped with the previous, maximally, 28 days.

\subsection{Feature Discussion}
\subsubsection*{What is a Routine?}
Most people will go on vacation during the year, which means the place where they sleep changes. In general, peoples' habits will inevitably change somewhat over time, and if one compares the routine of a certain person now to what their routine looked like a year ago, it is not unlikely to be very different. However just because a user changes their routine over time, does not mean they don't currently possess one. Therefore it was chosen to base the \textit{Routine Index} was chosen to be calculated based on the last 4 weeks of data in order to base the routine overlap on more recent days. An issue which was not dealt with is the fact that the routine on weekdays differs a lot from the routine during the weekend. This is especially true for people who spent 8 or more hours at work during the weekdays and spent those 8 hours somewhere else during Saturday and Sunday since it means the \textit{Routine Index} cannot exceed $\frac{2}{3}$ due to a third of the day's total hours being spent at a different place than usual. To add to this, even weekdays may look slightly different from one another, especially for those who are part of sports clubs which meet during certain days of the week. In future work it would be interesting to investigate whether or not comparing Mondays to Mondays and vice versa for every day in the week would yield more accurate results.

\subsubsection*{Real-Time Routine Index}
If the features have to be evaluated at any point of the day, as is the case for real-time computation then the \texit{Routine Index} cannot rely on a full day of data. To make the feature represent something meaningful in real-time it would have to reflect the routine of the user up until the current time of days, i.e. if calculated at 14:00 then it should only use the first 14 rows of the matrix. This means the \textit{Routine Index} may be high early in the day since people usually sleep the same place, but are open to deviating as the day progresses. This can be useful to an application programmer in a recommender-system setting, where a trigger based on the \textit{Routine Index} could be set, such that the user could be alerted when the value falls below a certain threshold.

\subsubsection*{Relying on Historical Data}
In order to calculate the \textit{Routine Index} we need to save and load the historical data somehow. Two approaches were considered.\\

The first approach involves keeping historical stops saved on disk. By doing so, the \textit{Places} can be found by clustering the stops with DBSCAN, and an \text{Hour Matrix} for each day in the last 4 weeks can then be computed and the average \text{Hour Matrix} can be derived from these. Afterward the \textit{Routine Index} can be calculated as the overlap between these two, as previously described. In the field study, the author had just over 70 stops per week, which corresponds to around 300 stops for a 4 week period, which is a very manageable number of elements to cluster with DBSCAN, which would be the only bottleneck in this approach.\\

The second approach involves keeping storing the \text{Hour Matrix} for each day on disk. However for this approach the \textit{Places} would also need to be saved such that places collected today could be compared (wrt. distance) to historical places. The reason this is necessary is to ensure that the matrix dimensions of all the matrices are the same and that these columns refer to the same \textit{Place}.  This approach has the potential to be computationally cheaper but is much more complex to implement and was therefore not chosen in this iteration.


