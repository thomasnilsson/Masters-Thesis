\section{Feature Descriptions}
The Mobility Features which were used were a subset of the features discussed by Saeb et al. \cite{Saeb2015} and Canzian et al. \cite{Canzian2015}. In addition a set of, what we will refer to as \textit{intermediate features} were also used which from the work of Cuttone et al. \cite{sparse-location-2014}. 

\subsection{Intermediate Features}
Common for many of the algorithms for finding user mobility features is that they rely on clustering of data points, in order to find the number of places. However when dealing with large amounts of data points it may be necessary to reduce the initial amount of data points such that these clustering algorithms are able to run faster. This down-sampling process will be carried out by clustering raw data points into what we shall refer to as \textit{Stops} indicating locations where the participant did not move around a lot. The \textit{Stop} notion was developed by Jonas Busk and is based on the work by Cuttone et al. \cite{sparse-location-2014}. The pre-processing produces \textit{intermediate features} from which the final mobility features are derived. We define these \textit{intermediate features} as follows:

\subsubsection*{Stop}
A collection of GPS points which together represent a visit at a known \texit{Place} (see below) for an extended period of time. A \textit{Stop} is defined by a location that represents the centroid of a collection of data points, from which a \textit{Stop} is created. In addition a \textit{Stop} also has an \textit{arrival}- and a \textit{departure} time-stamp, representing when the user arrived at the place and when the user left the place. From the arrival- and departure timestamps of the \textit{Stop} the duration can be computed.

\subsubsection*{Place}
A group of stops that were clustered by the DBSCAN algorithm developed by Ester et al. \cite{density-based-1996}. From the cluster of stops, the centroid of the stops can be found, i.e. the center location. In addition, it can be computed how long a user has visited a given place by summing over the duration of all the stops at that place.

\subsubsection*{Move}
The travel between two Stops, which the user will pass though a path of GPS points. The distance of a Move can be computed as the sum of using the haversine distance of this path. Given the distance travelled as well as departure and arrival timestamp from the Stops, the average speed at which the user traveled can be derived. 

\subsubsection*{Hour Matrix}
The Hour Matrix is an auxiliary feature used to compute the \textit{Home Stay} and \textit{Routine Index} feature. A matrix with 24 rows, each row representing an hour in a day, and columns equal to the number of places. The \textit{Hour Matrix} represents the time-place distribution for the user during a day.
\begin{figure}
    \centering
    \begin{tabular}{|l|l|l|l|l|}
    \hline
    \textbf{}        & \textbf{Place \#1} & \textbf{Place \#2} & \textbf{...} & \textbf{Place \#N} \\ \hline
    \textbf{00 - 01} &                    &                    &              &                    \\ \hline
    \textbf{01 - 02} &                    &                    &              &                    \\ \hline
    \textbf{...}     &                    &                    &              &                    \\ \hline
    \textbf{16 - 17} &                    &                    &              &                    \\ \hline
    \textbf{17 - 18} &                    &                    &              &                    \\ \hline
    \textbf{18 - 19} &                    &                    &              &                    \\ \hline
    \textbf{...}     &                    &                    &              &                    \\ \hline
    \textbf{23 - 00} &                    &                    &              &                    \\ \hline
    \end{tabular}
    \caption{An \textit{Hour Matrix}}
    \label{fig:time-table}
\end{figure}

\subsection{Feature Descriptions}
The features derived from this class are \textit{Home Stay (\%)}, \textit{Location Variance}, \textit{Number of Places}, \textit{Entropy} \textit{Normalized Entropy} , \textit{Distance Travelled} and the \textit{Routine Index (\%)}. These features are computed daily which for example means Home Stay refers to the portion of the time today which was spent at home. Below a short definition of each feature will be given since some of them differ somewhat from the previous work. A more detailed description will be given in Section \ref{section:definitions}.

\subsubsection*{Home Stay}
The portion (percentage) of the total time elapsed since midnight which was spent at home. Elapsed time is calculated from the departure time of the last known stop.

\subsubsection*{Location Variance}
The statistical variance in the latitude- and longitudinal coordinates.

\subsubsection*{Number of Places}
The number of places visited today.

\subsubsection*{Entropy}
The entropy with respect to time spent at places.

\subsubsection*{Normalized Entropy}
The normalized entropy with respect to time spent at places.

\subsubsection*{Distance Travelled}
The total distance travelled today (in meters), i.e. not limited to walking or running.

\subsubsection*{Routine Index}
The percentage of today that overlapped with the previous, maximally, 28 days.