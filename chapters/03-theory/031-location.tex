\section{Location Data}
The Navbar Global Positioning System (GPS) is a space-based radio-navigation system developed by The US Office of the Department of Defense \cite{gps-navstar} that provides location data to GPS receivers such as the one found in smart-phones. GPS is capable of delivering information such as latitude and longitude coordinates which indicates where on the Earth's surface the receiver is located, in addition to the altitude, i.e. the distance from the surface. Previously one would have to use a stand-alone GPS receiver as it was done by Ashbrook et al.  in 2002 \cite{using_gps_to_learn_significant_locations} in order to collect GPS data.\\

Nowadays, smart-phones contain GPS receivers which enable users to use a variety of navigation services and applications with their phone alone. For calculating distances between points, a standard Euclidean distance metric cannot be used since the earth is not a plane. Multiple methods of calculating the distance between GPS coordinates exist, one of the fastest to compute being the Haversine formula \cite{haversine-formula}. The Haversine distance computes the great circle distance between two points on a sphere. The Earth is however not exactly spherical, and therefore the Haversine distance is an approximation that works for shorter distances. Given the radius of a sphere $r$ and two points on the sphere, $A$ and $B$, the \textit{haversine distance} between the two points can be directly computed as 

\begin{equation}
\label{eq:haversine}
d = 2r \cdot \arcsin \Bigg( \sqrt{\sin^2 \bigg( \frac{lat_B - lat_A}{2} \bigg) + \cos(lat_A) \cdot \cos(lat_B) \cdot \sin^2 \bigg(\frac{ lon_B - lon_A}{2} \bigg)}\Bigg)
\end{equation}

