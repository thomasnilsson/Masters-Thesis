\section{Location Data}
The Navbar Global Positioning System (GPS) is a space-based radio-navigation system developed by The US Office of the Department of Defense \cite{gps-navstar} that provides location data to GPS receivers such as the one found in smart-phones. GPS is capable to delivering information such as latitude and longitude coordinates which indicates where on the Earth's surface the receiver is located, in addition to the altitude, i.e. the distance from the surface. Previously one would have to use a stand-alone GPS tracker as was done by Ashbrook and Starner \cite{using_gps_to_learn_significant_locations} in order to collect this data, but nowadays smart-phone contain GPS receivers which enables users to use a variety of navigation services and applications with their phone alone. From a developer perspective,  tracking location data on the Android- or iOS platform works by using an Application Programming Interface (API) which uses the GPS receiver inside the phone to place it on the Earth geographically. This API allows the streaming of location data in a continuous fashion which is used for navigation such as with Google Maps, where the user's location changes continually and it is important that the user is notified if they have to make a turn in due time. 

\subsection{Measuring Distance}
As mentioned, location data is defined by a latitude- and longitude coordinate which map to a point on the globe, however since the earth is a sphere and not a plane, a standard Euclidean distance metric cannot be used to calculate the distance between two points on the surface of the Earth. A few methods of calculating the distance between GPS coordinates exist, one of the fastest to compute being the Haversine formula \cite{haversine-formula} which calculates the great-circle distance between the two points on a sphere. The Earth is however not exactly spherical, and therefore the Haversine distance is just an approximation which works well when the distance has to be calculated many times, such as when calculating the distance of a path consisting of many points close to each other.

Given the radius of a sphere $r$ and two points on the sphere, $A$ and $B$, the \textit{haversine distance} between the two points can be directly computed as 
$$d = 2r \cdot \arcsin \Bigg( \sqrt{\sin^2 \bigg( \frac{lat_B - lat_A}{2} \bigg) + \cos(lat_A) \cdot \cos(lat_B) \cdot \sin^2 \bigg(\frac{ lon_B - lon_A}{2} \bigg)}\Bigg)$$
