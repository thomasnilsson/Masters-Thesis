\section{Future Work}
The application development process led to many of the improvements of the package API reflected in Chapter \ref{chapter:04} and \ref{chapter:05} however, a couple of issues remain both in terms of the API and the feature accuracy. In addition it is the plan to integrate the package into existing projects, such as CAMS \cite{CAMS} and MUBS \cite{mubs-rohani}.

\subsection{Data Collection and Parameter Tuning}
The field study resulted in a dataset of 2.5M data points which means there would have been an opportunity for tuning the parameters if time allowed. Parameter tuning would increase the accuracy of the features produced by the algorithms and thereby increase its usefulness. Some features vary greatly in their accuracy from participant to participant and it is unknown whether that is down to the commitment of the participants and how diligent they were in tracking their location, as well as their answering, or whether the algorithms simply do not consider certain commonly occurring edge cases. Not all of the days are annotated as previously discussed, far from it, however, a researcher can go by the days manually one by one for each participant and label certain of the features such as places visited, and to some extent, the Home Stay and Routine Index, although the latter two would require a great effort of fine-combing the data-sets. If feature-labeling was performed then it would simply be an optimization task where the optimal parameters for Stops and Moves are the parameters and the error of accuracy the objective to minimize. Also, to improve the validity of the answers given by participants in a future study, a short introduction should be given which defines what counts as a place, and how the routine question should be answered. If the answers are more valid and consistent, it is much easier to use them for evaluation of the features.

\subsection{Routine Index}
An issue that was not considered for this thesis is the fact that the routine on weekdays differs a lot from the routine during the weekend. This is especially true for people who spent 8 or more hours at work during the weekdays and spent those 8 hours somewhere else during Saturday and Sunday since it means the \textit{Routine Index} cannot exceed $\frac{2}{3}$ due to a third of the day's total hours being spent at a different place than usual. To add to this, even weekdays may look slightly different from one another, especially for those who are part of sports clubs which meet during certain days of the week. In future work, it would be interesting to investigate whether or not comparing Mondays to Mondays and vice versa for every day in the week would yield more accurate results. In addition, the \texit{Routine Index} should not rely on a full day of data if it is to be calculated in real-time. To make the feature work best on an incomplete day of data, it should reflect the routine of the user up until the current time of the day. This means if it is calculated at 14:00 then it should only take into consideration the data from the first 14 hours from previous days as well. This would result in the feature likely being very high early in the day, since people usually sleep the same place, but then may get lower as the day progresses. This varying routine index can be useful in real-time context, where triggers can be set up based on the \textit{Routine Index}, ex to alert the user when the value exceeds a certain threshold a certain threshold. As mentioned in section \label{sub:routine-index} it would also be highly relevant to incorporate moves into the Routine Index, such that commutes will be taken into account when computing the feature.

\subsection{Forced Daily Computation}
Currently, the implementation throws away Location Samples from previous days when computing the MobilityContext for today. This approach assumes that any data left from a previous date has been transformed into Stops and Moves, and therefore no longer is needed. However does not consider the case where a large part of the stored Location Samples have not been used, due to no computation having taken place. In some cases, whole days of Location Samples may end up being thrown away without Stops and Moves being computed from this data. The current way to avoid this is to compute features every day, making sure at least one computation takes place late in the evening such that minimal data is lost. Another way around it currently is to override the date, it is known that computation did not take place for a given date. This 'latest date of computation' can be kept track of by the programmer, but goes back to the problem of managing complexity. Ideally, this is done by the package itself, and can be solved the following way:
When Location Samples are loaded, group them by date, and compute Stops and Moves for each of these dates. Save the Stops and Moves, and then throw away the samples from these previous dates. out.

\subsection{Asynchronous Computation}
The asynchronous computation is cumbersome to set up and takes over 30 lines of code to perform. This should ideally be moved inside the package in the next iteration. Another improvement to make is not relying on lazy evaluation, as discussed in Chapter \ref{chapter:05}. In the current version, only the intermediate Features will be computed in the background thread, whereas all the derived features contained in the MobilityContext object returned by the asynchronous computation will not yet have been computed upon returning from the background thread. This is due to \textit{lazy evaluation}, which ensures that the features are not unnecessarily computed in advance, and only computed once they are requested. This could potentially mean freezing the UI thread since the features computed via lazy evaluation will be computed once the Mobility Context object is inside the main thread. The fix for this is to evaluate all features in the constructor of the Mobility Context class, which will imply longer guaranteed computation time for creating a Mobility Context object but will also guarantee that all features have been pre-computed in the background thread and will not block the UI thread. 

\subsection{Example Application}
The Pub package manager requires packages to have an example application to demonstrate its usage. Since the study application used an old version of the package API and does not display data, it should probably not be used further. Instead, the old version of the study app displayed in \ref{fig:app-features-screen} is a good candidate for an example app since it presents the calculated features to the user and can be implemented dynamically were features are constantly recomputed and updated.

\subsection{Integration and Maintenance}
The package fits into the \textit{CARP Mobile Sensing Framework} developed by CAHCET, as previously mentioned and will, therefore, continue to exist beyond this thesis. An integration into CAMS was not made as part of this thesis due to time constraints and the scope of the thesis. For the foreseeable future, it will be maintained by the author, who will be employed at CACHET as a research assistant. In additional, the MUBS recommender system by Rohani et al. \cite{mubs-rohani} is a smart-phone application used for treatment of bi-polar patients through recommendation of pleasant activities. The system does so by tracking patients' prior engaged activities and which the patients rate through the app manually. By using the mobility features we aim to add mental state and behaviour prediction to improve the recommendation algorithm, with these features being automatically generated.