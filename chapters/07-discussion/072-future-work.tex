\clearpage
\section{Future Work}
The application development process led to many of the improvements of the package API reflected in Chapter \ref{chapter:04} and \ref{chapter:05} however, a couple of issues remain both in terms of the API and the feature accuracy. In addition it is the plan to integrate the package into existing projects, such as CAMS \cite{CAMS} and MUBS \cite{mubs-rohani}.

\subsection{Missing Data}
As discussed prior in this chapter, this are substantial non-uniform gaps in the collected location data, even for the author. This is a common occurrence for 'in the wild' studies \cite{palmius2017}. Many variables can influence the background location tracking in smartphones, such as the availability of GPS signal, battery, as well as available memory on the phone. The author had a relatively new phone (iPhone XS) and was diligent in making sure the phone was tracking the location as often as possible, yet still has significant gaps in his data. This means other participants that were less diligent or had older phones have even more gaps is still missing in certain periods. \\

Feature values can be influenced to a high degree by missing data, especially the Routine Index. Currently the Routine Index feature ignores periods of missing data when computing the overlap between the routine matrix and today's hour matrix, however a lot of gaps in the data will result in the feature overshooting. The solution for this missing data problem is either to use a dedicated GPS receiver, or to use an imputation strategy for missing data. Imputation is most likely the best solution, and can be carried out by filling out gaps using the saved stops and moves as historical data. This may likely bias the features however it will be a promising alternative to the current method.

\subsection{Data Collection and Parameter Tuning}
The field study resulted in a dataset of 2.5M data points which means there would have been an opportunity for tuning the parameters if time allowed. Parameter tuning would increase the accuracy of the features produced by the algorithms and thereby increase its usefulness. Some features vary greatly in their accuracy from participant to participant and it is unknown whether that is down to the commitment of the participants and how diligent they were in tracking their location, as well as their answering, or whether the algorithms simply do not consider certain commonly occurring edge cases.\\

Many days are without a subjective answer meaning they have been thrown away in the data analysis. Rather than relying on participant answers, a researcher could in theory manually label each participants data sets with appropriate feature values. Certain features such as the number of places visited will be easier to manually detect, compared to to other features such as entropy. If this was done, then finding optimal parameters for the algorithms would be a machine learning task. \\

Regarding the study, to improve the validity of the answers given by participants in a future study, a short introduction should be given to each participant prior to starting the study. The introduction should define what counts as a place, what counts as their home, and how a routine is defined. This was not done for the field study in this thesis which is likely a cause for a lowered quality of the subjective answers. If the participants are instructed beforehand, one can hope that answers become more consistent and the results of the data analysis will be more informative.

\subsection{Routine Index}
An issue that was not considered for this thesis is the fact that the routine on weekdays differs a lot from the routine during the weekend. This is especially true for people who spent 8 or more hours at work during the weekdays and spent those 8 hours somewhere else during Saturday and Sunday. For such a person the \textit{Routine Index} cannot exceed $\frac{2}{3}$ due to a third of the hours in a day being spent at a different place than their usual place. Even weekdays may look slightly different from one another, especially for those who are part of sports clubs which meet during certain days of the week. In future work, it would be relevant to examine whether comparing Mondays to Mondays, Tuesdays to Tuesdays, etc. would yield a more accurate Routine Index.\\

In addition, the \texit{Routine Index} should not rely on a full day of data if it is to be calculated in real-time. To make the feature work best on an incomplete day of data, it should reflect the routine of the user up until the current time of the day. This means if it is calculated at 14:00 then it should only take into consideration the data from the first 14 hours from previous days as well. This would result in the feature likely being very high early in the day, since people usually sleep the same place, but then may get lower as the day progresses. This varying routine index can be useful in real-time context, where triggers can be set up based on the \textit{Routine Index}, ex to alert the user when the value exceeds a certain threshold a certain threshold.\\

As mentioned in section \label{sub:routine-index} it would also be highly relevant to incorporate moves into the Routine Index, such that commutes will be taken into account when computing the feature. It not very clear how this could be represented with the existing definition of the Hour Matrix, however one possibility is making an Hour Matrix for moves only, showing which travels were taken at which time. The routine index could then be computed from both the \textit{stop hour matrix} and the \textit{move hour matrix}.

\subsection{Forced Daily Computation}
Currently, the implementation throws away Location Samples from previous days when computing the MobilityContext for today. This approach assumes that any data left from a previous date has been transformed into stops and moves, and therefore no longer is needed. However does not consider the case where a large part of the stored Location Samples have not been used, due to no computation having taken place. In some cases, whole days of Location Samples may end up being thrown away without stops and moves being computed from this data. The current way to avoid this is to compute features every day, making sure at least one computation takes place late in the evening such that minimal data is lost. Another way around it is to override the date, it is known that computation did not take place for a given date. This 'latest date of computation' can be kept track of by the programmer, but goes back to the problem of managing complexity. \\

Ideally, this is done by the package itself, and can be solved by first grouping location samples by date when loaded. Next, stops and moves are computed for each of the dates. Lastly, the stops and moves are saved to the disk and all the location samples from prior days can be thrown away.

\subsection{Asynchronous Computation}
The asynchronous computation is cumbersome to set up and takes over 30 lines of code to perform. This should ideally be moved inside the package in the next iteration. Another improvement to make is not relying on lazy evaluation, as discussed in Chapter \ref{chapter:05}. In the current version, only the stops, moves and places are be computed in the background thread. All remaining features are computed inside the MobilityContext object which have not yet been evaluated at the time the object is constructed, due to lazy evaluation. This means the remaining features are computed synchronously upon request, which will likely happen in the main thread. This could lead to freezing the UI-thread, and the fix for this is to compute all features in the constructor of the Mobility Context class. The trade-off will be that it takes longer to compute the features in the asynchronous call, but there will no need to compute the features in the main thread.

\subsection{Example Application}
The Dart package manager, Pub, requires packages to have an example application to demonstrate its usage. Since the study application used an old version of the package API and does not display data, it should probably not be used further. Instead, the old version of the study app displayed in \ref{fig:app-features-screen} is a good candidate for an example app since it presents the calculated features to the user and can be implemented dynamically were features are constantly recomputed and updated.

\subsection{Integration and Maintenance}
The package fits into the \textit{CARP Mobile Sensing Framework} developed by CACHET, as previously mentioned and will, therefore, continue to exist beyond this thesis. An integration into CAMS was not made as part of this thesis due to time constraints and the scope of the thesis. The package will be maintained by the author, who will be employed at CACHET as a research assistant. In additional, the MUBS recommender system by Rohani et al. \cite{mubs-rohani} is a smart-phone application used for treatment of bi-polar patients through recommendation of pleasant activities. The system does so by tracking patients' prior engaged activities and which the patients rate through the app manually. By using the mobility features we aim to add mental state and behaviour prediction to improve the recommendation algorithm, with these features being automatically generated.