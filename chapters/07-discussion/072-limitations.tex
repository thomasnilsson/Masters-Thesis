\section{Limitations and Improvements}
The package developed is in a functioning state which can produce useful features, however there are certain improvements to be made from here. Improvements have been made during the time in which the package was used for developing the app which mostly pertain the the API, and not to the implementation of the features. 

\subsection{Overall API Minimalism}
The API was made drastically simpler in the time after developing the application. Very important lessons were made in developing in the application and a lot of the logic which had to be written in the application, by an application programmer was moved inside the package.


\subsection{Forcing Daily Computation}
Force daily calculation of features, otherwise stops and moves are not saved. Strike a balance between saving too much raw data on the phone and not losing data. Concrete suggestion, save location samples, and when computing features, group data by date, and compute stops and moves for each date. When a date has been calculated, delete all raw location samples for that day.

\subsection{Saving Data Manually}
Firstly, regarding the package itself, the API is still relies on the user saving the data via the serialization interface. It is not currently clear how this can be simplified for the application programmer without baking in the location collection into the package, however this may be the only option. 

\subsection{Asynchronous Computation}
Currently, the application programmer has to manually specify to use Dart Isolates in order to compute the Intermediate Features and the Mobility Context in a background thread. However once again a lot of responsibility is put on the application programmer which could instead be moved inside the package. This makes the package much easier to use, at the expense of introducing complexity, which means the library grows larger and gets harder to maintain.
If this is implemented, then Lazy evaluation should probably not be used, instead compute on initialization since initialization is async. There is no point in doing some of the computation in the background thread, only to having to do the remaining feature computation in the UI thread, which for a large dataset, i.e. 28 days of Stops and Moves can prove computationally intensive thus freezing the UI.

\subsection{Feature Accuracy}
The dataset contains 2.5M data points which means there would have been an opportunity for tuning the parameters if time allowed. Parameter tuning would increase the accuracy of the features produced by the algorithms and thereby increase its usefulness. Some features vary greatly in their accuracy from participant to participant and it is unknown whether that is down to the commitment of the participants and how diligent they were in tracking their location, as well as their answering, or whether the algorithms simply do not consider certain commonly occurring edge cases. Not all of the days are annotated as previously discussed, far from it, however a researcher can go by the days manually one by one for each participant and label certain of the features such as places visited, and to some extent the Home Stay and Routine Index, although the latter two would require a great effort of fine-combing the the data-sets. If feature-labeling was performed then it would simply be an optimization task where the optimal parameters for Stops and Moves are the parameters and the error of accuracy the objective to minimize. 
