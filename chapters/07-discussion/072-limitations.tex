\section{Limitations and Improvements}
The package developed is in a functioning state which can produce useful features, however there are certain improvements to be made from here. Firstly, regarding the package itself, the API is quite cumbersome to use when calculating the Routine Index, since it relies on historical data, and historical data requires it to be stored on the device and loaded when necessary. This introduces complexity to the package. It is currently unclear precisely what the preferred state of the API would be; one the one side it is currently very open and flexible to use, but it also takes many lines of code to produce a Mobility Context with prior Contexts. The alternative would be to close off the API such that storing and loading is done internally and the application programmer simply has to collect and store the data, but not the intermediate features. 

A very minimalistic API taken to the extreme could look like the following:

\begin{minted}{dart}
    MobilityContext mc = MobilityContext(prior: true);
\end{minted}

This requires the package to perform all the saving and loading internally. This could for example be put in the constructor of the MobilityContext class:

\begin{minted}{dart}
    MobilityContext({this.prior: true}) {
    _timestamp = DateTime.now();
    
    }
\end{minted}

\subsection{Asynchronous Computation}
Currently, the application programmer has to manually specify to use Dart Isolates in order to compute the Intermediate Features and the Mobility Context in a background thread. However once again a lot of responsibility is put on the application programmer which could instead be moved inside the package. This makes the package much easier to use, at the expense of introducing complexity, which means the library grows larger and gets harder to maintain.

\subsection{Feature Accuracy}
The dataset contains 2.5M data points which means there would have been an opportunity for tuning the parameters if time allowed. Parameter tuning would increase the accuracy of the features produced by the algorithms and thereby increase its usefulness. Some features vary greatly in their accuracy from participant to participant and it is unknown whether that is down to the commitment of the participants and how diligent they were in tracking their location, as well as their answering, or whether the algorithms simply do not consider certain commonly occurring edge cases. Not all of the days are annotated as previously discussed, far from it, however a researcher can go by the days manually one by one for each participant and label certain of the features such as places visited, and to some extent the Home Stay and Routine Index, although the latter two would require a great effort of fine-combing the the data-sets. If feature-labeling was performed then it would simply be an optimization task where the optimal parameters for Stops and Moves are the parameters and the error of accuracy the objective to minimize. 
