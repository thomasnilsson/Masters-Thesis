\section{Technology}

\subsection{Tracking Location Data}
Location data sampled on a mobile device contains information such the geo-spatial position on the globe, as well as a time-stamp. Location data may also include information such as altitude, and other attributes such as the accuracy with with the sample was made and the speed at which the receiver was moving at the time of the sample being taken. Tracking location data on the different mobile platforms Android and iOS work similarly; a Location API is available to the application programmer which allows the application to stream location given a set of parameters such as the frequency of sampling as well as how much movement should be detected before a new sample is taken. The battery is often drained if the frequency is too high and normally would be taken into consideration, however the question of whether a low sampling rate would work was not part of the thesis, so the sampling rate was set to max. Ethically speaking, location data is very sensitive and with a very high sampling rate provides a high level of insight, down to the second in fact, of where the user was during the day. For these two reasons the location data should not leave the phone in practice and the processing should take place on the phone. Instead, only the mobility features which contain less sensitive information are sent to a database, such as in a study setting. The location data should also be tracked in the background since the user cannot be expected to not use their phone at all during the time they are being tracked, since this is virtually all the time. For this to work, certain permissions and flags must be configured correctly on the respective platforms such that the application using this package can sample location data and process it in the background. Shortly after the start of this thesis, Google limited how the Android Location API handles background updates, which meant that an Android application must be in the foreground to track location such as is the case with Google Maps navigation where it is very explicit\footnote{\url{https://developer.android.com/training/location/background}}. Due to time constraints however, a good solution to this problem was not found. 

\subsection{Flutter Plugins and Packages}
The Flutter Framework is a cross-platform development framework that was released by Google in 2018. Flutter uses the Dart programming language (also made by Google) and makes it possible to write a codebase purely in Dart, and compiling this code to native iOS and Android code. What this means is that an application developer can create an application for both platforms with a single codebase. However whenever native-specific APIs have to be invoked such as that of the camera or various phone-sensors, a package developer has to write a library which does so, by means of a method invocation from the Dart language which triggers the corresponding API invocation ob the Android or iOS platform, which is corresponds to the Java/Kotlin and Objective-C/Swift programming languages. This method invocation library is referred to as a \textit{plugin} within the Flutter world, in contrast to a \textit{package} which simply invokes other Dart code and as such contains no platform-specific source code. The \textit{Mobility Features Package} will be a Flutter package which contains a collection of algorithms which provides an application programmer with object-oriented abstractions which allows him/her to calculate relevant features for a mobile health application. The Location API, available on both iOS and Android, will not be invoked directly from this package, since that would require it to be a plugin. This has two main upsides: From the point of the application developer, it allows him/her to use their location plugin of choice (of which there are many \footnote{\url{https://pub.dev/packages?q=location}} with specific parameters for how location is tracked (ex frequency and distance). Secondly, from the perspective of the maintainer of this package, the package becomes much more modular and simpler by design and in turn easier to maintain.

\subsection{Flutter and Cross Platform}
Write in the Dart programming language\\
Compile to native code\\
Sensors and other API's not necessarily mirrored\\
UI and most general things implemented by the Google team\\