\section{Package Design and API}
The software package is intended to be used by a Flutter developer and has to have an \textit{Application Programming Interface} (API) that strikes the balance between being easy to use and hiding complexity from the programmer. This can be done in a numerous ways, but given that Flutter uses the object-oriented programming language Dart, it makes sense to encapsulate complexity using Classes rather than pure functions. With programming languages such as Python which are not inherently object-oriented, the APIs often expose a series of functions which return basic objects such as tuples of floats, integers and strings, rather than objects to the application programmer.

\subsection{Including Intermediate Features}
Since the \textit{Stops} need to be saved manually on the device, it was chosen to give access to pre-processing features such as \textit{Stops}, \textit{Moves} and \textit{Places} since they are also needed to make \textit{Mobility Context}. It may also be the case that they are very useful on their own for compressing raw GPS data into a much smaller and more coarse-grained dataset which captures where the user was stationary. This can be very useful if combined with reverse geo-coding in which the geo-locations of \textit{Places} are converted to addresses, which can then be further transformed into a place category such as school, work, grocery story, sports club and much more, which can give insights into what the user is actually spending their time on. Reverse geo-coding will however not be part of this thesis. The intermediate features do however contain very sensitive information, and probably more sensitive that raw GPS data because the exact places the user was at and moved between is stored, and the noisy data is removed. In a future iteration these intermediate features will likely either be contained in their own Flutter package, which the Mobility Features Package depends upon, and then hidden away in the Mobility Features Package such that not sensitive information may escape the package.

\subsection{Features}
The main output to the application developer will be a list \textit{Mobility Context} objects which have the described mobility features as fields. The \textit{Mobility Context} class takes in Date, as well as a List of Stops, Moves and Places, where the Stops and Moves are on the given Date, and the Places are from a given period, i.e. from today and as far back as has been tracked with the maximum being 28 days prior. 

Rather than using both Places and Stops for instantiation, the data model could be modified such that a Place contained a list of Stops made at that place. This would mean grouping the Stops by Place rather than Date, and would require filtering to take place every time a \textit{Mobility Context} object is created, to remove all Stops not on that specific date. In a real-world scenario the application developer will usually have the SLPs for the current day available, and from those the Stops today can be generated which means no filtering is required. Grouping Stops into places would however lead to a nicer data model, but a design choice was made in favor of less computation. 

\subsection{Saving Historical Data}
It was chosen to include a serialization API in the package since the algorithms rely on using historical data to compute the Routine Index. Otherwise the programmer would have to save this themselves, which is not very user friendly. While the serialization interface was provided in order to easily support the saving and loading of SLPs, Stops and Moves it is however not automatic. It was left up to the application programmer to perform this, since it is only necessary if the \textit{Routine Index}is desired and will make the data collection and computation pipeline much more complex. However in the future work it would make sense to use a real database on the phone instead, which supports querying objects using their Date field. 