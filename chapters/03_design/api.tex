\section{Package Design and API}
The software package is intended to be used by a Flutter developer and has to have an \textit{Application Programming Interface} (API) that strikes the balance between being easy to use and hiding complexity from the programmer. This can be done in a numerous ways, but given that Flutter uses the object-oriented programming language Dart, it makes sense to encapsulate complexity using Classes rather than pure functions. With programming languages such as Python which are not inherently object-oriented, the APIs often expose a series of functions which return basic objects such as tuples of floats, ints and strings, rather than objects to the application programmer.

\subsection{Pre-processing Features}
Since the Stops need to be saved manually on the device
It was chosen to give access to pre-processing features such as Stops, Moves and Places since these are very useful in on their own. Stops can be useful for compressing a dataset of raw GPS data into a much smaller and more coarse-grained dataset which captures where the user was stationary. Moves give an indication of accurate distances between stops and as such can be used to calculate distance travelled as well as velocity, while still compressing the dataset by a factor of over a thousand. Places can be used to pinpoint the main clusters the user stay at during the day. This can be very useful if combined with reverse geo-coding in which the geo-location of a place is converted to an address, which can then be further transformed into a place category such as school, work, grocery story, sports club and much more, which can give insights into what the user is actually spending their time on. The latter will not be part of this thesis though.

\subsection{Features}
The main content of the package is the Mobility features derived from the pre-processing features. 

\subsection{Serialization}
It was chosen to include a serialization API in the package since the algorithms rely on using historical data to compute the Routine Index. Otherwise the programmer would have to save this themselves, which is not very user friendly.

A serialization interface was provided in order to easily support the saving and loading of SLPs, Stops and Moves. However this saving and loading was chosen not to be automatic and it was left to the application programmer to perform this, since it is only necesary if the routine index is to be calculated. The other option would be to make it non-optional, however this would both make the package much more complex and computationally hinder the performance of computing features for a single day. The Routine Index is likely not relevant for all applications and as such the API was left in a state where it supports calculating the Routine Index, but does not make it compulsory. 