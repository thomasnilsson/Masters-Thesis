\section{Field Study}
A small-scale study was conducted which ran for 3 weeks and included 10 participants (including the author). In the study the participants used the application discussed in Chapter \ref{chapter:05-api-implementation} that collected their location data and computed mobility features daily. In addition to this the application also had a diary consisting of 4 questions that the user had to fill out each day. In order to make it easy for the user to remember filling out the diary, a FireBase Cloud Messaging service was created which sent the user a daily notification at 8 PM. The time 8PM was chosen due to being relatively late while still being early enough in the day that people would still be checking their phone. Some people go to bed at 9-10 PM which had to be taken into account. The diary questions were related to 3 of the Mobility Features which were \textit{Number of Clusters}, \textit{Home Stay} and \section{Routine Index}. The point of the questions were to get a subjective estimates of the values of these features. It is important to stress the fact that these these answers are estimates since the user cannot be expected to give very precise answers and secondly they are very subjective since the definition of things such as a 'Place' may vary a lot from person to person.

\subsection{Choice of Parameters}
To choose the most important parameters 

Questions:

\begin{itemize}
    \item How many unique places (including home) did you stay at today?
    \item How many hours did you spend away from home today? (Rounded-up)
    \item Did you spend time at places today that you don't normally visit?
    \item On a scale of 0-5, how much did today look like the previous, recent days? (Where 0 means 'not at all' little and 5 means 'Exactly the same')
   
\end{itemize}

Corona crisis made it uncommon for people to move around a lot and most of the time was spent at home. This was not optimal conditions for testing the algorithms, but not impossible at all.

\subsection{Data Storage}
While development of the application was still on-going the data collected on the phone had to be stored somewhere online in order to verify the output of the algorithms run in Python as well as on the phone. For this a FireBase file storage server was used which supports file uploads. A JSON file was uploaded each day, with the upload being triggered manually. 


