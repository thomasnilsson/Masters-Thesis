\section{Inferring User State from Location Data}


% Using location data to infer behaviour and mental state
% Litterature review should indicate which features are relevant
% Feature are anything derived from the location data, i.e. clusters, trajectories, number of clusters etc.


% Saeb
%   More related to mental health
%   Can we infer users mental state from location data
%   A list of mobility features provided, which this thesis to large extent is based upon
%   Does not discuss how the Circadian Movement feature is computed exactly
%   It was not defined in a way which allowed it to be directly replicated
%   PHQ9 questionnaire
%   Our study suggests that it is possible to monitor depression passively using phone sensor data, and in particular, GPS. This has significant public health implications. Most people are unwilling to answer questions repeatedly over long periods of time, while passive monitoring could improve the management of depression in populations, allowing at risk patients to be treated more quickly as symptoms emerge, or monitoring patients’ responses during treatment.

The work done by Saeb et al. \cite{Saeb2015}, Canzian et al. \cite{Canzian2015}  and to some extend Cuttone et al. \cite{sparse-location-2014} forms the basis of this thesis with regard to mobility features. The work done by Ashbrook et al. \cite{learning_significant_locations} is not directly related but is one of the pioneering contributions within the field.\\

% \paragraph{Saeb et al.} 
Saeb et al. provides a list of mobility features and relate them to mental state and user behaviour, the featueres are outlined in Table \ref{tab:saeb-features}. Out of these features the authors found that Home Stay, Location Variance, Normalized Entropy and Circadian movement correlate strongly with the PHQ9 score (see Appendix \ref{appendix:questionnaires}). Clinical assessment of depressive symptoms is usually done using questionnaires, such as the patient health questionnaire (PHQ-9). These questionnaires evaluate symptoms related to the patients state of mind, which for depressed patients tend to change. Saeb et al. finds a strong correlation between certain mobility features and the PHQ9 score obtained by the patients in a field study. Their findings therefore indicate that it is possible to monitor depression passively using GPS data tracked via the smart-phone. They explain that most people are unwilling to answer questions repeatedly over long periods of time, which is the case with an evaluation system based on the PHQ9 questionnaire. Passive monitoring via the patients smartphone could great improve the management of depression in populations by discovering relapses of high-risk patients earlier, so that they may receive the treatment they need before it is too late. \\

% \paragraph{Canzian et al.} 
Canzian et al. provides a contribution similar to Saeb et al. and additionally discusses how existing digital systems for diagnosing depression require the user to interact with the device. These interactions can be inputs such as mood-state provided a few times a day, which is highly subjective and thus error-prone. Canzian et al. addresses this issue via objective patient data, namely the mobility patterns of users. These patterns were derived from location data gathering through smartphone apps, which as Canzian et al. puts it, is unobtrusive monitoring. These patterns are described by the mobility metrics (i.e. features) listed in Table \ref{tab:canzian-features}. The authors found a significant correlation between these metrics and depressive moods similar to \cite{Saeb2015} for which the \textit{PHQ-8} questionnaire is used as  reference. This is the same questionnaire as the PHQ9, although only containing the first 8 questions. While Canzian et al. and Saeb et al. describe many of the same features, Canzian provides much more detailed definitions of the features. A good example of this is the \textit{Routine Index} feature, which loosely corresponds to the \textit{Circadian Rhythm} feature in \cite{Saeb2015}. Canzian et al. \cite{Canzian2015} uses the notion of trajectories including \textit{places} and \textit{moves} defined by Spaccapietra et al. \cite{trajectories2008} to define the \textit{mobility trace} feature. This feature is used to derive the rest of the features, which is also the approach used for this thesis, although the terminology differs slightly.\\

% \paragraph{Cuttone et al.} 
Cuttone et al. \cite{sparse-location-2014} investigates how accurate significant locations can be inferred from sparsely sampled location data using the GPS tracker in smartphones. For longitudinal studies it can be a problem for the phone battery if the sampling rate is too high. Therefore a more low-energy approach is used where the sampling rate- and the location accuracy are low. A small-scale study was conducted in which location data collected continuously and the users filled out an online diary on a daily basis. The aim was to compute features \textit{Stops} and \textit{Places of Interest (POI)} for each user and match them with the provided answers. These features are described in Table \ref{tab:features-cuttone}. \\

\begin{table}[]
    \centering
\begin{tabular}{|p{0.4\textwidth}|p{0.5\textwidth}|}
\hline
{\ul \textbf{Feature}}       & {\ul \textbf{Description}}                                                                                                                                        \\ \hline
Place of Interest (POI)                        & \textit{Location of relevance frequented by the user, e.g. home, gym or workplace}                                                                                                                    \\ \hline
Stop            & \textit{A visit to a POI with an arrival- and departure timestamp}                                                                                  \\ \hline
\end{tabular}
    \caption{The Mobility Features defined by Cuttone et al. \cite{sparse-location-2014}}
    \label{tab:features-cuttone}
\end{table}


\begin{table}[h]
    \centering
\begin{tabular}{|p{0.4\textwidth}|p{0.5\textwidth}|}
\hline
{\ul \textbf{Feature}}      & {\ul \textbf{Description}}                                                                                                                     \\ \hline
Number of Clusters          & \textit{Number of clusters found by the clustering algorithm}                                       \\ \hline
Home Stay                   & \textit{Percentage of time stayed at the home cluster}                                              \\ \hline
Location Variance           & \textit{Variance of the location of stationary points}                                              \\ \hline
Location Entropy            & \textit{Entropy of the time distribution wrt. clusters}                                             \\ \hline
Normalized Location Entropy & \textit{Entropy normalized by the maximum possible entropy. Invariant to the number of clusters} \\ \hline
Transition Time             & \textit{Percentage of time not stationary}                                                                                                  \\ \hline
Total Distance              & \textit{Total distance travelled}                                                                                                              \\ \hline
Circadian Movement          & \textit{The degree to which changes in the location follow a 24 hour rhythm}                        \\ \hline
\end{tabular}
    \caption{The Mobility Features defined by Saeb et al. \cite{Saeb2015}}
    \label{tab:saeb-features}
\end{table}



\begin{table}[h]
    \centering
\begin{tabular}{|p{0.4\textwidth}|p{0.5\textwidth}|}
\hline
{\ul \textbf{Feature}}       & {\ul \textbf{Description}}                                                                                                                                        \\ \hline
Place                        & \textit{A location the user frequently visits}                                                                                                                    \\ \hline
Significant Place            & \textit{The top 10 most visited places out of all places}                                                                                  \\ \hline
Move                         & \textit{A trajectory between two places}                                                                                                                          \\ \hline
Mobility Trace               & \textit{The stops and moves a given time interval}                                                                                                                \\ \hline
Total Distance               & \textit{Distance travelled for a given time interval}                                                                                                             \\ \hline
Max Distance                 & \textit{Maximum distance between any two points for a given time interval}                                                                                        \\ \hline
Radius of Gyration           & \textit{Radius of the area covered for a given time interval}                                                                                                     \\ \hline
Standard deviation of Displacement         & \textit{Standard deviation for stop locations for a given time interval}                                                                                          \\ \hline
Max Distance from Home       & \textit{Maximum distance from home to any place visited in a given time interval}                                                                                 \\ \hline
Number of Places             & \textit{The number of unique places visisted for a given time interval}                                                                                           \\ \hline
Number of Significant Places & \textit{The number of unique significant places visited for a given time interval}                                                                                \\ \hline
Routine Index                & \textit{The average difference between the mobility behaviour of the user for a given interval, on a given day, compared to the same time interval on other days} \\ \hline
\end{tabular}
    \caption{The Mobility Features defined by Canzian et al. \cite{Canzian2015}}
    \label{tab:canzian-features}
\end{table}


The work by Ashbrook et al. \cite{learning_significant_locations} from 2002 is one of the earliest contributions to the field of GPS data based user context. Here, user context refers to where the user will move next, given their current location. While this is not related to mental health necessarily, it is interesting to look into the approach taken given that it is a pioneering contribution. The features defined by Ashbrook are location features only, i.e. features consisting of a latitude and longitude and are shown in Table \ref{tab:features-ashbrook}. 



\begin{table}[]
    \centering
\begin{tabular}{|p{0.4\textwidth}|p{0.5\textwidth}|}
\hline
\textbf{Feature} & \textbf{Description}                                                               \\ \hline
Place            & \textit{A geo-location at which the GPS signal could not reach such as a building} \\ \hline
Location         & \textit{The centroid of a cluster of places}                                       \\ \hline
Sublocation      & \textit{A location within a location, i.e. university building on a uni-campus}    \\ \hline
\end{tabular}
    \caption{The Mobility Features defined by Ashbrook et al. \cite{learning_significant_locations}}
    \label{tab:features-ashbrook}
\end{table}


\subsection{Data Collection}
% Canzian
% \paragraph{Canzian et al.} 
Canzian et al. built a custom Android app (MoodTraces) which collected data from 46 users over almost 10 months. The app had participants fill out a survey daily between 16:00 and 02:00 the subsequent day, where a phone notification prompted the participant to fill out the survey. Each survey contained a 'trap' question which asked whether the user was current at home or work. The answer to this question could be verified in the data analysis phase, and was used to filter out invalid surveys. The survey asked the user to name all the depressive symptoms they had on that day. PHQ8 score is normally done by asking every 14 days, and have the participant answer how many times they had a symptom over the last 14 days. By using the questionnaires, the PHQ8 scores could be computed by summing the amount of days with the symptoms, which in total add up to a score. It would happen that users forgot to fill out their questionnaire or gave an invalid survey response, in which case an interpolation method was used to cover gaps in the data.\\

% Saeb
% \paragraph{Saeb et al.} 
Saeb et al. used a custom Android application, Purple Robot, which collected the users' GPS data with a sampling rate set to once every 5 minutes. Participants had the application manually installed on their existing phones or were given an Android phone with the app pre-installed on it which they were to use as their primary phone. Participants filled out a PHQ9 questionnaire just prior to starting the study, and once again once the study concluded after 14 days.\\

% Ashbrook
% \paragraph{Ashbrook et al.} 
Ashbrook et al. wrote their contribution in 2002 where old-fashioned GPS receivers were the standard tool for collection location data. Sampling was done once per second which provides very high resolution, however GPS signals had less than ideal coverage in 2002 which meant that the receiver did not work indoors most of the time - which actually worked to their advantage since it made identifying places easier. In addition the GPS receiver also did not capture data if the user was moving at a speed of less than 1 kilometer per hour, to conserve battery. Six users participated in a 2002 study which lasted for seven months.\\

% Cuttone
% \paragraph{Cuttone et al.} 
Cuttone et al. conducted a study with 7 participants including the author, with the author being tracked for four months and the other participants being tracked for two months. Data collection was done with a custom smartphone app developed using the Funf Open Sensing framework \footnote{\url{https://www.funf.org/}}, where participants were provided an Android phone to use as their main phone. Sampling was carried out with a very low sampling rate of once were 15 minutes which had the upside of not draining the phone battery much. Participants were also instructed to fill out a daily diary of the place they had been at, and their movements. This was done through an online spreadsheet and participants were reminded via email. The authors comment that using diaries as ground truth is difficult due to participation compliance decreasing over time, since filling out a diary is tedious. In addition the concepts of stops and POIs are somewhat ambiguous and subjective which means that even if participants do fill out their diary, it is subjective and error-prone. This was evident in the study where participants would sometimes fill out sequences of POIs in the wrong temporal order, and entries were prone to have typos in them which later had to be manually corrected by the researcher. At the end of the study the participants were asked to create a subjective list of their POIs with coordinates obtained through Google Maps. \\

\subsection{Computing Features}
%Canzian features
\paragraph{Canzian et al.} uses an algorithm for sampling location data in a way which saves phone battery but also helps identify places. For this, the accelerometer is used by an activity recognition algorithm to determine the user's activity state. This is done by modelling the user as being a state-machine with 3 states: 
\begin{itemize}
    \item Static (S)
    \item Moving (M)
    \item Undecided (U)
\end{itemize}

A user is said to arrive at a place if the transition $M \rightarrow U \rightarrow S$ happens and a departure is noted when $S \rightarrow U \rightarrow M$ happens. By using all the location samples in between the arrival and departure, the centroid of the place can be found using a clustering algorithm. In addition, an iterative clustering algorithm is used to merge places with close proximity to each other, specifically a distance of 200 meters is used as the merging criterion. \\ 

% Saeb features
% \paragraph{Saeb et al.} 
Saeb et al. uses a two-stage pipeline for location data processing: 
\begin{itemize}
    \item Each data point is labeled as either being in a stationary or transitional state by using the speed of each data point. Points are labeled using a threshold of 1 km/h; a speed lower than the threshold indicates the state is stationary and higher is a transitional point. 
    \item The K-means clustering algorithm is used to group the \textit{stationary points} into frequently visited places. Since the number of places, i.e. the number of clusters for the K-means algorithm (the parameter \textit{K}) is not known beforehand, the best parameter value is found using cross-validation.
\end{itemize}

After this pre-processing procedure was applied, the features described in Table \ref{tab:saeb-features} were derived.\\ 

% Ashbrook Features
% \paragraph{Ashbrook et al.} 
Ashbrook et al. used the fact than the GPS receiver could not get a signal inside most buildings which was used to identify a user's \textit{places}. The GPS receiver used had a 15 meter accuracy meaning the location coordinates reported may be prone to noise. To remove noise the \textit{K-means} clustering algorithm was applied to the data which resulted in the \textit{locations} features. Inside \textit{locations}, a set of \textit{sub-locations} can be found, which are nodes in a network of small-scale locations. A prime example of a location with sub-locations is a university campus, where the sub-locations are the different department buildings. These features are listed in Table \ref{tab:features-ashbrook}. \\ 

% Cuttone Features
% \paragraph{Cuttone et al.} 
Cuttone et al. used two main approaches for finding stops and POIs. The first in voles finding stops first and then grouping stops into POIs. The second approach works the other way around and identifies POIs first by clustering the location data, and infers the stops from these POIs afterward. For finding the stops, 2 different algorithms, Distance Grouping and Speed Thresholding, were tried each yielding similar results.

\begin{itemize}
    \item Distance grouping involves iterating location points temporally and grouping them based on a max-distance measure. Each stop is created with a single location point $loc_{i}$ and each subsequent location point $loc_{i+k}$ is then added to the stop, with $k$ being incremented by 1 each iteration. This is done until $distance(loc_{i}, loc_{i+k} > d_{max}$ where $d_{max}$ is the maximum allowed distance. By increasing the max-distance fewer stops of longer duration will be found. For finding the optimal value of this parameter, the $f_1$ score was calculated for the specific dataset.  

    \item Speed thresholding determines which location points belonged to stops, and which points did not. The max-speed parameter was determined by grouping data samples based on their timestamp, each group having a duration of $T$. Choosing the max-speed parameter was then done by considering the speed between the median location of each bin. Gaussian Mixture Modelling was also tried which looks at all the location samples independent of their timestamp, and identifies POIs directly from clusters with large densities. 
\end{itemize}

For both these algorithms, the DBSCAN clustering algorithm by Ester et al. \cite{density-based-1996} was applied afterwards, which groups stops into places and marks noisy stops as not belonging to a place.

The second approach used a \textit{Gaussian Mixture Model} to cluster the raw location data into places, and infer the stops from these clusters. The upside to this method is that it is much simpler to perform on paper, but the downside is that it does not consider the temporal dimension at all which can contain a lot of useful information. This means the GMM uses more compute power than the other two approaches. 

For this thesis, a modified version of the distance grouping algorithm was chosen is described in Chapter \ref{chapter:03}.