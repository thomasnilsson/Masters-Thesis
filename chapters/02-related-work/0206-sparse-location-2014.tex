\section{Inferring Human Mobility from Sparse Low Accuracy Mobile Sensing Data}
Human Mobility is relevant for a number of different applications such \textit{disease spread,} \textit{urban planning}, \textit{managing traffic }as well as understanding \textit{social interactions}. Previous contributions within these fields have been using Call Detail Records (CDR) which is the art of triangulating the user's location by use of cell-phone towers. These sources produce very coarse-grained estimates wrt. location and time. For this paper a more traditional location data sampling was used, using the GPS tracker in smartphones, however, for longitudinal studies it can be a problem for the phone battery if the sampling rate is too high. Therefore a more low-energy approach is used where the sampling rate- and the location accuracy are low. 

\subsection{Data Collection and Features}
A study with 6 participants was conducted in which location data collected continuously and the users filled out an online diary on a daily basis. Given a series of temporally-ordered data points for a user, the aim was to compute features called \textit{Stops} and \textit{Places of Interest (POI)}. A \textit{POI} is defined as a location of high relevance such as that person's school, gym or a supermarket as is denoted by a type $POI = (ID, lat, lon)$ and a \textit{Stop} at a POI is defined as the period of time in which the user stayed at a POI with a given ID, i.e. $Stop = (arrival, departure, ID)$. To find Stops, both a \textit{Gaussian Mixture Model}, as well as the \textit{DBSCAN} algorithm, were applied for clustering. To evaluate the best model, the $f_1$ score of each model was computed, and in the end, it was found that both algorithms performed similarly for each of the participants. 