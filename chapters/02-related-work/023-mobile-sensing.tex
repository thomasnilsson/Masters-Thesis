\section{Mobile Sensing Frameworks}
\subsection{AWARE}
The \textit{AWARE Framework} \cite{aware2015} is an open-source toolkit and a reusable platform for capturing, inferring, and generating user-contexts on mobile devices. Phones possess high-quality sensors but are resource-constrained with regards to their processing speed and battery capacity, which must be considered when computing contexts in real-time. The \textit{AWARE Framework} therefore aims to ensure an easy way of collection contexts for the application developer, and it is demonstrated how the tool can reduce the software development effort of researchers when building mobile tools for developing context-aware apps and doing so with minimal battery impact. By designing an API that conceals the underlying implementation of sensor data-retrieval, and exposes an abstract representation of a \textit{user context object}, AWARE shifts the focus from software development to data collection and analysis. Currently AWARE is available on both iOS and Android and supports a number of data channels \footnote{\url{https://awareframework.com/sensors/}} such as the built-in sensors, as well as \textit{Application Usage}, \textit{SMS}, and \textit{Phone Call Logs}. Some of the channels are however not available on iOS due to the iOS developer API, in general, being more restrictive than that of its Android counterpart.

\subsection{CARP Mobile Sensing}
Inspired by AWARE, the CARP Mobile Sensing (CAMS) Framework \cite{CAMS}, developed by CACHET, is a mobile sensing framework for adding digital phenotyping capabilities to a mobile-health app. CAMS is designed to collect research-quality sensor data from the many smartphone data channels such as sensors and location data, in addition to external sensors that the phone is connected to, such as wearable devices. The main focus of the framework is to allow application programmers to design and implement a custom mobile health app without having to start from scratch, with regards to the sensor integration, by enabling the programmers to add an array of mobile sensing capabilities in a flexible and simple manner. This would include adding support for collecting health-related data channels such as \textit{ECG}, \textit{GPS}, \textit{Sleep}, \textit{Activity}, \textit{Step Count} and much more. Additionally, to format the resulting data according to standardized health data formats (like \textit{Open mHealth} schemas \footnote{\url{https://www.openmhealth.org/documentation/#/schema-docs/schema-library}}. Last but not least, the collected data should be uploaded to a server, using an API (such as \textit{REST}), and should come in a standardized format such that it may easily be imported for data analysis. To include as many data channels as possible the application should also be able to support different wearable devices for ECG monitoring and activity tracking. Hence, the focus is on software engineering support in terms of a solid programming API and a runtime execution environment, that is being maintained as the underlying mobile phone operating systems and APIs are evolving. Moreover, the focus is on providing an extensible API and runtime environment, which together allow for adding application-specific data sampling, wearable devices, data formatting, data management, and data uploading functionality to an application. In addition, in order to reduce the number of integrations needed, the framework must be cross-platform such that it supports both Android and iOS, without having a codebase for each platform. An integration for the \textit{Mobility Features Package} into the CAMS Framework is planned but is a future goal beyond this thesis.