\section{Mobile Sensing Frameworks}
The \textit{AWARE Framework} \cite{aware2015} is an open-source toolkit and a reusable platform for collecting data and generating user-context on mobile devices. Phones possess high-quality sensors but are resource-constrained with regards to their processing speed and battery capacity, which must be considered when computing contexts in real-time. The \textit{AWARE Framework} aims to ensure an easy way of collection contexts for the application developer. More specifically it aims to reduce the software development effort of researchers when building mobile tools for developing context-aware apps. By designing an API that conceals the underlying implementation of sensor data-retrieval, and exposes an abstract representation of a \textit{user context object}, AWARE shifts the focus from software development to data collection and analysis. Currently AWARE is available on both iOS and Android natively, meaning two code bases will have to be maintained if one wishes to write both an Android and iOS application. AWARE supports a number of data channels \footnote{\url{https://awareframework.com/sensors/}} such as the built-in sensors, as well as more HCI-based data sources such as \textit{Application Usage}, \textit{SMS}, and \textit{Phone Call Logs}. Some of the channels are however not available on iOS due to the iOS developer API being more restrictive than that of its Android counterpart for most data channels.\\

Inspired by AWARE, the CARP Mobile Sensing (CAMS) Framework by Bardram \cite{CAMS} is a mobile sensing framework for adding digital phenotyping capabilities to a mobile-health app. A number of technological platforms for mobile sensing have been presented over the years and a lot of knowledge on how to facilitate mobile sensing has been accumulated. CAMS is a modern cross-platform software architecture providing a reactive and unified programming model that emphasizes extensibility, maintainability, and adaptability. CAMS is written in Flutter and in contrast to AWARE uses a single code base to compile to both Android and iOS.\\

CAMS is designed to collect research-quality sensor data from the many smartphone data channels such as sensors and location data, in addition to external sensors that the phone is connected to, such as wearable devices. The main focus of the framework is to allow application programmers to design and implement a custom mobile health app without having to start from scratch, with regards to the sensor integration. This is done by enabling the programmers to easily add mobile sensing data channels to their application. This would include adding support for collecting health-related data channels such as \textit{ECG}, \textit{GPS}, \textit{Sleep}, \textit{Activity}, \textit{Step Count} and many more. Additionally, to format the resulting data according to standardized health data formats (like \textit{Open mHealth} schemas \footnote{\url{https://www.openmhealth.org/documentation/#/schema-docs/schema-library}}). Last but not least, the collected data should be uploaded to a server, using an API (such as \textit{REST}), and should come in a standardized format such that it may easily be imported for data analysis. To include as many data channels as possible the application should also be able to support different wearable devices for ECG monitoring and activity tracking. Hence, the focus is on software engineering support by providing a high level programming API and a run-time execution environment. In order to simplify the process for the researcher further, the researcher only has to maintain a single code base - in contrast to AWARE. This is because CAMS  is written in the cross-platform framework flutter where one common code-base in the Dart programming language is used. Maintenance of the framework will be ongoing, and is required for it to stay relevant as the underlying mobile phone operating systems and APIs are evolving. \\

An integration for the \textit{Mobility Features Package} into the CAMS Framework is planned but is a future goal beyond this thesis (see Future Work in \ref{chapter:07}).