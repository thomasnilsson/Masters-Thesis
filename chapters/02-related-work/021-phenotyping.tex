\section{Mobile Sensing and Digital Phenotyping}
The work by Insel \cite{digital_phenotyping} deals with the topic of collecting and aggregating user data into a so-called \textit{digital phenotype}. It is predicted that by 2050, the biggest impact in psychiatry and mental health will have been the revolution in technology- and information science. Smartphones have become ubiquitous in the past decade and there are over three billion smartphones with a data plan worldwide, each of which has computing power which surpasses supercomputers of the 1990s. In areas around the world without access to resources such as clean water and food, ownership of a smartphone and access to information has become a symbol of modernity. \\

In the realm of psychiatry, a current data collection problem is the dependence on self-reporting of sleep, appetite, and emotional state, even though it is recognized that depression will impair people's ability to remain objective in assessing their own behavior and thus data is prone to be faulty. Another problem discussed by Insel is how people suffering from mental illness tend to not seek help before it is too late. Depressive relapses are therefore also often reported with considerable delay for patients currently in treatment. The smartphone offers an objective form of mental-state measurement which uses built-in sensors such as geo-location, accelerometer, and human-computer interactions (HCI) to infer the state of the patient. This makes it possible to assess people by using data in a real-time fashion, rather than in retrospect as is currently is done. Digital phenotyping could in theory fill the role of a smoke detector which provides early signs of relapse and recovery, without replacing the face-to-face consultations entirely. In addition, this also allows researchers to track patients in their own environment, rather than in a clinical environment. \\

Palmius et al. \cite{palmius2017} explain that early identification and intervention for emergent mood episodes is highly important for managing mood disorders. Educating the patients to be able to self-identify early symptoms of relapsing has been associated with improvements in patient's ability to recover from relapses. Ubiquitous mobile computing has made it possible to develop low-cost, long-term solutions for mood state and behaviour monitoring. Early contributions focused on self-reporting of daily symptoms which suffers from decreased patient engagement over time, and when the patient relapses. Passively monitoring patient through background data collection via the smart-phone require no manual input from the patient is therefore a promising venue.