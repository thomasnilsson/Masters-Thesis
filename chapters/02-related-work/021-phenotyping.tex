\section{Digital Phenotyping}

Mobile sensing is the field of using the phone's sensors to collect data user data and can include predicting the user state from this data. Some applications of mobile sensing include mental health monitoring, for which different terms are used such as "digital phenotyping" and "context generation". Mobile sensing for mental healthcare has the potential of providing a practical and low-cost approach to deliver psychological interventions for the prevention of mental health disorders \cite{mobile-based-interventions} as well as bringing mental healthcare to populations that do not have access to traditional psychotherapy \cite{future-mental-health}. Within the context of mobile sensing for behaviour and mental state, location data is commonly used. 

The paper \cite{digital_phenotyping} deals with the topic of collecting and aggregating user data into a so-called digital phenotype. It is predicted that by 2050, the biggest impact in psychiatry and mental health will have been the revolution in technology- and information science. Smartphones have become ubiquitous in the past decade and there are over three billion smartphones with a data plan worldwide, each of which has computing power which surpasses supercomputers of the 1990s. In areas around the world without easy access to clean water, ownership of a smartphone and by proxy, rapid access to information has become a symbol of modernity. 

In the realm of psychiatry, a current data collection problem is the dependence on self-reporting of sleep, appetite, and emotional state, even though it is recognized that depression will impair people's ability to remain objective in assessing their own behavior and thus data is prone to be faulty \cite{digital_phenotyping}. Another current problem is how it how people suffering from mental illness tend to not seek help before it is too late. Depressive relapses are therefore also often reported with considerable delay for patients currently in treatment. 

The smartphone offers an objective form of mental-state measurement which is referred to was \textit{Digital Phenotyping}, which uses built-in sensors such as geo-location, accelerometer, and human-computer interactions (HCI) to infer the state of the patient. This makes it possible to assess people by using data in a real-time fashion, rather than in retrospect as is currently is done. Digital phenotyping could in theory fill the role of a smoke detector which provides early signs of relapse and recovery, without replacing the face-to-face consultations entirely. In addition, this also allows researchers to track patients in their own environment, rather than in a clinical environment.
