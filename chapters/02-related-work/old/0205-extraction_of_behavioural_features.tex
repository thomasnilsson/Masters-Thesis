\section{Extraction of Behavioral Features from Smartphone and Wearable Data}
Published in 2019 \cite{extraction-of-behavioural-features}, this paper focuses on many channels for collecting user data, with each channel resulting in a set of features. Examples of channels are GPS data, phone usage, step count, etc. The upside of having many channels is improved accuracy since more data is available to describe the behavior of the user, and if certain channels are inactive during certain periods of time, the other channels can cover the missing data, and thus periods of missing data will be avoided with high likelihood. 

The paper uses the \textit{AWARE Framework }\cite{aware2015} to collect data such as  \textit{Bluetooth}, \textit{Call-log}, \textit{Location}, \textit{Campus Map}, \textit{Phone Usage}, \textit{Step Count} and \textit{Sleep State} (via FitBit).

\subsection{ High-Level Derived Features}
The daily behavior of the user is predicted through four high-level features extracted from the data channels.\\

\textbf{User mobility}: The movement patterns of the user are derived from location tracking and the campus map feature.\\ 

\textbf{Communication Patterns}: Derived from call log and text messages.\\

\textbf{Mobility and communication}: Derived from Bluetooth device IDs within range, i.e. Bluetooth scan.\\

\textbf{Physical activity:} Derived from step-count during different windows of the day, i.e. when the user is walking a lot.\\

\subsection{Time windows}
The day (24 hours) is split into 4-time windows, which helps identify where the home is since the user likely sleeps the same place most days of the week, which is likely during the \textit{night} time-window.

\begin{itemize}
\item Night (00-06)
\item Morning (06-12)
\item Afternoon (12-18)
\item Evening (18-00)
\end{itemize}

\subsection{Location Features}
GPS coordinates collected throughout the day have a time stamp connected to them, which makes it possible to calculate a list of interesting features within a day of the user moving around. The features are similar to those found in \cite{Saeb2015} adds two additional features, namely \textit{Radius of gyration} and \textit{Time spent at top3 clusters}. However a couple of additional contributions are found. Doryab et al. defines a \textit{bout} as a small window of time, which can be defined based on the granularity wanted, i.e. 5 minutes. Furthermore, a clear definition for the \textit{home} cluster is given as being everything within 10 meters of the center of the home location center. Time spent at home is defined at the time spent within 100 meters of the home centroid. The activity of studying is defined as spending 30 minutes or more in an academic building, while sedentary, which is taking fewer than 10 steps per \textit{bout}.

\subsection{Bluetooth Features}
From the Bluetooth Scan, the devices found each has a unique ID and can, therefore, be categorized into the following groups:

\begin{itemize}
    \item Self, i.e. the user (the device scanned most often)
    \item Related, ex-colleagues, flatmates (scanned less often)
    \item Other, other people (scanned least often)
\end{itemize}

The three categories are created through K-means clustering, by initially creating two clusters ($K=2$), i.e. self and others. Secondly another model is fitted to the data with $K=3$ clusters, and the model which fits the data the best is chosen, i.e. through cross-validation or BIC.


\subsection{Phone Usage}
Phone usage is based on the screen state which at any time can be only one of the following: On, off, lock, unlock. Tracking the state of the screen together with the timestamp at which the state changes, a few features can be calculated, which are:

\begin{itemize}
    \item Number of unlocks per min
    \item Time spent interacting with the phone
    \item Total time unlocked
    \item Hour of the day of first unlock/turned on
    \item Hour of the day of last unlock, lock, turned off
    \item Time spent interacting with the phone (sum of time between unlocking and off/lock)
    \item Standard deviation of time spent interacting with the phone
\end{itemize}


\subsection{Sleep Features}
By tracking heart rate, and other signals with a FitBit watch, the Fitbit API gives access to the \textit{sleep state} of the user, which at any time is one of the following: Asleep, restless, awake, unknown.

A number of samples will be made during the night and thus the simplest features which can be created are the number of samples in each of the four states. 
From this, two higher-level features are computed, which are: \\

Weak Sleep Efficiency: $WSE = \frac{n_{asleep} + n_{restless}}{n_{asleep} + n_{restless} + n_{awake}}$\\

Strong Sleep Efficiency: $SSE = \frac{n_{asleep}}{n_{asleep} + n_{restless} + n_{awake}}$\\

With $n_{state}$ being the number of samples for that state.


\subsection{Steps Features}
From the step-count channel features pertaining to the fine-grained activity, without the location, for a whole day, can be calculated:

\begin{itemize}
    \item Number of steps
    \item Max steps in a \textit{bout}
    \item Number of active \textit{bouts}
    \item Min, Max, Avg. length of sequence active \textit{bouts} in a row
\end{itemize}
    
\subsection{Behavioural Change Tracking}

GPS coordinates collected throughout the day have a time stamp connected to them, which makes it possible to calculate a list of interesting features within a day of the user moving around. The features are similar to those found in \cite{Saeb2015} adds two additional features, namely \textit{Radius of gyration} and \textit{Time spent at top3 clusters}. However a couple of additional contributions are found. Doryab et al. defines a \textit{bout} as a small window of time, which can be defined based on the granularity wanted, i.e. 5 minutes. Furthermore, a clear definition for the \textit{home} cluster is given as being everything within 10 meters of the center of the home location center. Time spent at home is defined at the time spent within 100 meters of the home centroid. The activity of studying is defined as spending 30 minutes or more in an academic building, while sedentary, which is taking fewer than 10 steps per \textit{bout}.




The change in the user's behavior over a longer time period, i.e. a few weeks can be tracked by using a number of the proposed features, and is done by applying a simple linear regression as well as a piecewise linear regression (with two lines) to the dataset. The calculate behavioral change features are then the following:

\begin{itemize}
    \item Derivatives ($a_i$) of the linear model: $y=a_1 x_1 + a_2 x_2 + ... + a_n x_n + c$.
    \item Change in slope (derivative) for each behavioral feature
\end{itemize}

Given data for $n \in \mathbb{Z}$ timesteps, i.e. weeks, choose some breakpoint $m < n \in \mathbb{Z}$ and create the piecewiese linear function:

\[ \begin{cases} 
      a_1 x_1 + a_2 x_2 + ... + a_n x_n + c & t <    m \\
      b_1 x_1 + b_2 x_2 + ... + b_n x_n + d & m \leq n 
   \end{cases}
\]

The two-line segments are then fitted to the data with a feature selection algorithm. The resulting slope vectors for the selected features are then behavioral features.

%%%%%%%%%%%%%%%%%%%%%%%%%%%%%%%%%%%%%%%%%%%%%%%%%%%%%%%%%%%%%%%%%%%%%%

