\section{Thesis Overview}
% The chapter often ends with an overview of the remainder of the thesis. This is usually done by summarizing each chapter as a brief paragraph. Good chapter descriptions will relate the contributions of that chapter to the particular goals of your \section{Thesis Overview}
% The chapter often ends with an overview of the remainder of the thesis. This is usually done by summarizing each chapter as a brief paragraph. Good chapter descriptions will relate the contributions of that chapter to the particular goals of your thesis.
This section will provide an overview of the thesis in the form of a short summary of each remaining chapter.

\subsection{Related Work}
In this chapter, the most relevant research with regards to ubiquitous computing and generating mobility features will be laid out. This includes relevant contributions within the broader context of mobile sensor context generation, algorithms for pre-processing spatial data, clustering GPS data, how features can be generated from geospatial clusters, and lastly what the features can be used for.

\subsection{Design}
Here, it will be discussed which requirements the final product should fulfill. This includes what should be computed and how, i.e. which algorithms were needed to produce the needed mobility features, and how some these algorithms can work in real-time. In addition, the data model design is discussed which has implications for how the underlying implementation looks like and to some extent, the public-facing API.

\subsection{Algorithm Implementation}
Here, it will be discussed how the off-line algorithms were concretely implemented both in Flutter such that they fulfill the requirements function in real-time, in a Flutter environment.

\subsection{API Design and Implementation}
Here, it will described how the API was designed and implemented from a Flutters-specific perspective. This includes which algorithms and data model objects were public facing to the user and which were left internal. Furthermore the CAMS framework will also be explained as well as how this packages integrates into CAMS.

\subsection{Validation}
Validation of the package will be discussed both from a quantitative- as well as a qualitative perspective. The quantitative validation will verify that the algorithms produce meaningful and correct results, which will be validated by means of unit testing as well as data analysis in which subjective data reported by participants will be compared to the features computed by the algorithms. The qualitative part will be much more focused on how the software package was designed, i.e. the ease of use for the application developer as well as design of the API. Concretely this can be evaluated in terms of the number of lines an application developer needs to write in order to use the package.

\subsection{Results and Discussion}
Here the results will be discussed as well as how one may go about developing the software package further, including changing the existing algorithms and which design- and implementation choices could have been made differently. This includes how the API is designed with regards to trade-offs between varying the abstraction-level, complexity and openness of the API.

\subsection{Conclusion}
The most important findings and things that could be worked on in the future will be described in this chapter.

