\section{Background}
% Some authors provide a miniature literature review to give the reader enough background to understand the context of the research; a full review is usually deferred to the second chapter of the thesis. Other authors actually do the full review in Chapter 1, but that is rarer.

% Some questions you should be able to answer in general terms after reading this section are:

% What is the research context and discipline that the thesis chapter fits within?

% In general, who has looked at this area before?
\subsection{Previous Work}
Clustering location data into places has been done previously in a more general context, for example to learn significant locations and predicting where the user will move next \cite{learning_significant_locations} or to learn whether or not low accuracy sampling can be used to identify these locations \cite{sparse-location-2014}. Later on, location data has been used within the context of depression, by calculating certain features such as the percentage of time spent at home \cite{Saeb2015, Canzian2015}. 

% In general, what other work complements this research?
In regards to Behavioural Activation, D. Rohani developed two recommender systems for treating depression while working on his PhD at CACHET \cite{mubs-rohani, moribus}, both of which rely on manual user input, and it is a future improvement to incorporate location data into the recommendation-algorithm.


% What is the motivation for studying a particular problem?


% What makes it worth the effort?

% Is it a 'real' problem in everyday life?

% Is it a 'theoretical' problem that is worth solving?