\section{Background}
% Some authors provide a miniature literature review to give the reader enough background to understand the context of the research; a full review is usually deferred to the second chapter of the thesis. Other authors actually do the full review in Chapter 1, but that is rarer.

% Some questions you should be able to answer in general terms after reading this section are:
% This section will provide a context for the thesis in the form of a short introduction to the topic of deression and secondly a brief litterature review within mobile sensing and its applications.

Major Depressive Disoder (MDD), commonly known as depression, is a serious illness that is resource-intensive to treat using traditional methods such as face to face consultations. This costs society a lot of money, and it is, therefore, worthwhile to explore treatments that can improve the quality of life of these patients. 

\subsection*{Treatment with Digital Phenotyping}
Among many treatments, Behavioural Activation is very well supported by mobile sensing technologies which have become a growing field due to the rise of ubiquitous computing. 

% Is it a 'real' problem in everyday life?
Currently, the diagnosis in bipolar disorder, also known as manic depression, relies on manual patient information and clinical evaluations and judgments with a lack of objective testing. Some of the essential clinical features of individuals suffering from bipolar disorder are changes to their daily behavior \cite{objective_smartphone_data_as_diagnostic_marker}. These changes in behaviour can be captured by the user's \textit{mobility context}, and can thus provide very helpful in customizing a Behavioural Activation treatment. A mobility context can be generated by user rules (i.e., manual), and triggered by events (i.e., system, user activity) from semi- or full- automated algorithms. Here, it will be investigated how the context can be generated in a fully-automatic, real-time fashion.

% In general, who has looked at this area before?
\subsection*{Previous Work}
Clustering location data into places has been done previously in a more general context, for example to learn significant locations and predicting where the user will move next \cite{learning_significant_locations} or to learn whether or not low accuracy sampling can be used to identify these locations \cite{sparse-location-2014}. Recently, location data has also been used within the context of depression, by calculating certain features such as the percentage of time spent at home \cite{Saeb2015, Canzian2015}. In addition \cite{Saeb2015} also showed that certain mobility features correlate strongly with scoring highly on the Patient Health Questionnaire (PHQ-9) (see Appendix \ref{appendix:questionnaires}), which indicates an individual is depressed.  

% In general, what other work complements this research?
In regards to Behavioural Activation, D. Rohani developed two recommender systems for treating depression while working on his PhD at CACHET \cite{mubs-rohani, moribus}, both of which rely on manual user input, and it is a future improvement to incorporate location data into the recommendation-algorithm. Lastly, the work done at CACHET on the CARP Mobile Sensing Framework\footnote{\url{https://github.com/cph-cachet/carp.sensing-flutter/wiki}} within mobile sensing provides an ecosystem in which the resulting package would fit in.