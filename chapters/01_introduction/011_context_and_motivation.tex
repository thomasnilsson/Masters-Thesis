\section{Context and Motivation}
% Almost every Chapter 1 begins with a section that sets the scene and that motivates the problem being studied. It describes some domain and indicates a problem in general terms.

% Some questions you should be able to answer after reading the motivation section are:

% What is the general area being addressed?
Most people walk around with a very powerful computer in their pocket, that contains a variety of sensors capable of painting a picture of the user's current context. A user's context is a set of digital characteristics, also called features, which arise from an individual interacting with their environment, such as when moving around between places in one's everyday life. This thesis will describe the design and implementation of a software package for the Flutter framework capable of generating a fully-automated mobility context in real-time, by tracking location data on a smart-phone. The package will be part of the Carp Mobile Sensing Framework (CAMS) which provides an array of digital phenotyping capabilities to an application developer.\\

\subsection{Depression and Digital Phenotyping}
% What is the motivation for studying a particular problem?
A field where context generation is highly useful is the treatment of mental illness where continuous patient monitoring can provide researchers with objective, concrete measurements from a patient's everyday life.

% What makes it worth the effort?
Depression is a serious illness and is resource-intensive to treat using traditional methods such as face to face consultations. This costs society a lot of money, and it is, therefore, worthwhile to explore treatments that can improve the quality of life of these patients. Among many treatments, Behavioural Activation is very well supported by mobile sensing technologies which have become a growing field due to the rise of ubiquitous computing. Mobility context can be a very helpful tool in customizing the Behavioural Activation treatment. Context can be generated by user rules (i.e., manual), and triggered by events (i.e., system, user activity) from semi- or full- automated algorithms. Here, it will be investigated how the context can be generated in a fully-automatic, real-time fashion.\\

% Is it a 'real' problem in everyday life?
Currently, the diagnosis in bipolar disorder, also known as manic depression, relies on manual patient information and clinical evaluations and judgments with a lack of objective testing. Some of the essential clinical features of individuals suffering from bipolar disorder are changes to their daily behavior \cite{objective_smartphone_data_as_diagnostic_marker}. In addition \cite{Saeb2015} showed that certain features that can be derived from GPS data correlate strongly with scoring highly on the Patient Health Questionnaire (PHQ-9), which indicates an individual is depressed.  In addition, features such as having little to no everyday-routine, of staying home excessively correlate strongly with being depressed. Therefore objective smartphone data reflecting behavioral activities are highly relevant for investigating this ailment.\\

\subsection{Challenges in Mobile Computing Research }
% Is it a 'theoretical' problem that is worth solving?
The research problem addressed in this project is focused on the software engineering aspect of developing a digital phenotyping library and relates to the \textit{re-usability of source code} as well as the current \textit{lack of support for real-time computation}. 

\subsubsection{Re-usability of Source Code}
Firstly, there have been numerous contributions within the field of using smartphone data for phenotyping and predicting the user's state for many years. However, much of the research conducted, specifically within the field of user-mobility context generation, has been done without publishing source code in a way such that it may be used by other researchers in the future. Mainly this comes down to researchers not publishing the source code at all, due to source code not being the focus of their research. Sometimes the source code is released, however, but then over time suffers from an (understandable) lack of maintenance and thus becomes obsolete. Furthermore, researchers tend to focus on only the Android platform since this is the more prevalent platform of choice in many countries, however, this is not a good research practice for reasons we will elaborate on later.

\subsubsection{Lack of Support for Real-time Computation}
In addition, the existing contributions gather all of the data before feature computation is carried out. This has the clear advantage of making it easier to run and evaluate the mobility feature algorithms. However, this also has the disadvantage that the feature generation is not possible in real-time, where they are the most useful. Certain features rely on prior contexts being computed and readily available and are therefore easily implemented when done in an off-line fashion on a desktop computer. However, in a real-time situation on a phone, these prior contexts will need to be stored on a device that has limited storage, and the algorithms must be adjusted to read in these stored set contexts. The proposed software package will solve the problem of implementing these algorithms from scratch as well as allow the application programmer to compute the user's mobility context in real-time.

\subsection{Who Would Benefit from this?}
% Would anyone care if I solved this?
Researchers within the field of mental health illnesses from a health-tech background will benefit greatly from using a Flutter package with a high level of abstraction, in which the researcher's focus is moved from software engineering to data analysis and study design. Currently, researchers often choose to start from scratch which is a laborious task that takes up valuable time which could be spent on more relevant work. Furthermore, since Flutter is capable of compiling to both iOS and Android source code, studies need not be restricted to a single mobile platform, hence why Flutter was the app-framework of choice.\\

An additional benefit of computing features on the phone itself, in contrast to computing it on a central server or desktop, is the anonymization of location information. The features themselves contain much less sensitive information and therefore has implications on privacy as well as making GDPR compliance easier.