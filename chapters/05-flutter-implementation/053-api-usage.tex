\section{Using the Package}
This section will be a mirror of the official instructions on how to use the package, as of version \textit{1.1.5}. For getting started, the programmer has to download the package by adding it as a dependency in their \verb|pubspec.yaml| file of the Flutter project and running the following command:

\begin{minted}{sh}
flutter packages get
\end{minted}

Once the dependency has been downloaded it can be imported as follows:

\begin{minted}{dart}
import 'package:mobility_features/mobility_features.dart';
\end{minted}

Features are computed using the 3 lines of code displayed in Figure \ref{fig:code-example-intro}:

\begin{figure}[h]
    \centering
    \begin{minted}{dart}
    /// Collect data with a location plugin
    List<LocationSample> locationSamples = ///

    /// Store data via the package
    await ContextGenerator.saveSamples(locationSamples);
    
    /// Compute Features via the package
    MobilityContext context = await ContextGenerator.generate();
    \end{minted}
    \caption{All the three lines of code necessary for the application developer to write}
    \label{fig:code-example-intro}
\end{figure}

The exact method for arriving at this stage is outlined in the following 4 steps.

\subsection*{Step 1: Collect location data}
Location data collection is, as mentioned, not part of the Mobility Features package, and the programmer will, therefore, have to a location plugin such as \url{https://pub.dev/packages/geolocator}. From here, each incoming location object has to be converted to a \verb|LocationSample| via the constructor

Below is shown an example where incoming \verb|Position| objects are coming in from the \verb|GeoLocator| plugin and are being handled after saved to a list, in the \verb|_onData()| call-back method.

\begin{minted}{dart}
List<LocationSample> locationSamples = [];
...

void _onData(Position d) async {
    GeoPosition geoPos = GeoPosition(d.latitude, d.longitude);
    LocationSample sample = LocationSample(geoPos, d.timestamp);
    locationSamples.add(sample);
}
\end{minted}

\subsubsection*{Step 2: Save location data}
The location data must be saved on the device such that it can be used in the future. Saving the data to persistent storage prevents it from being lost should the RAM reset. Given that the programmer has a collected the samples in a list, the samples can be serialized using the \verb|save()| method of the MobilitySerializer.

\begin{minted}{dart}
await ContextGenerator.saveSamples(locationSamples)
\end{minted}

Ideally, saving the data is done with a certain interval, such as every time 100 samples are collected. 

\subsection*{Step 3: Compute features}
The Features can be computed using the static class \verb|ContextGenerator| which uses the stored location samples, as well as historic features to compute the features for today.

There most basic computation is done as follows:
\begin{minted}{dart}
MobilityContext context = await ContextGenerator.generate();
\end{minted}

Note: it is not possible to instantiate a `MobilityContext` object directly. 
It must be instantiated through the `ContextGenerator.generate()` method.

\subsubsection*{Step 3.1: Compute features with prior contexts}
Should the programmer wish to compute the routine index feature as well, then prior contexts are needed. Concretely, the application needs to have tracked data for at least 2 days, to compute this feature. The generation of a Mobility Context using prior contexts is done using the \verb|usePriorContexts| argument and setting it to \verb|true|.

\begin{minted}{dart}
MobilityContext context = 
    await ContextGenerator.generate(usePriorContexts: true);
\end{minted}

\subsubsection*{Step 3.2: Compute features for a specific date}
By default, a \textit{MobilityContext} object uses the current date as a reference to filter and group data, however, should one wish to compute the features for a specific date, then it is possible to do so using the \verb|today| argument and providing the desired date.

\begin{minted}{dart}
DateTime myDate = DateTime(01, 01, 2020);
MobilityContext context = 
    await ContextGenerator.generate(today: myDate);
\end{minted}

\subsection*{Step 4: Get features}
All features are implemented as \textit{getters} for the MobilityContext class and can, therefore, be retrieved using the dot-notation. 

\begin{minted}{dart}
MobilityContext context = // Generate context

List<Place> places = context.places;
List<Stop> stops = context.stops;
List<Move> moves = context.moves;

int numberOfPlaces = context.numberOfPlaces;
double homeStay = context.homeStay;
double entropy = context.entropy;
double normalizedEntropy = context.normalizedEntropy;
double distanceTravelled = context.distanceTravelled;
double routineIndex = context.routineIndex;
\end{minted}