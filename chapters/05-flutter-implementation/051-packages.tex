\section{Flutter, Packages, and Plugins}
Flutter is a cross-platform app development framework developed by Google and released in 2018. It allows an application programmer to write a mobile application using a single codebase written in the Dart programming language, and compile this source code to a native Android and iOS application. This has the clear advantage of reducing the amount of labor needed to produce mobile applications which most of the time need to be released on both platforms. Packages and Plugins are the Flutter equivalent of a software library that is hosted on the Dart package manager at \url{pub.dev}, the Mobility Features Package is hosted at \url{pub.dev/packages/mobility_features}. The author has written several packages for the Flutter framework released under the CACHET publisher\footnote{\url{https://pub.dev/publishers/cachet.dk/packages}}. The Mobility Features Package has a library file of the same name as the package \textit{mobility\_features.dart} which is the central point of import statements for all the source code. All import statements are made within this file, and each file belonging to the library are declared using the \textit{part} keyword.

\begin{minted}{dart}
    library mobility_features;
    
    import 'dart:math';
    ...
    
    part 'mobility_functions.dart';
    ...
\end{minted}

Each file included in the library will have the equivalent \textit{part of} keyword at the top of their file, which allows the file to import all dependencies from the library, and makes the file public to other files within the library and vice versa.

\begin{minted}{dart}
    part of mobility_features;
    ...
\end{minted}

Packages also come with a set of unit tests. Unit testing is discussed in Appendix \ref{appendix:unit-testing} where selected tests are shown.