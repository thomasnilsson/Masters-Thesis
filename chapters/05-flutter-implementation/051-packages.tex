\section{Flutter Packages and Plugins}
When collecting user data which reflects the daily mobility patterns of the user, it is important to note that location data should be tracked in the background. Otherwise the phone cannot be used nce the user cannot be expected to not use their phone at all during the time they are being tracked since this is virtual all the time. For this to work, certain permissions and flags must be configured correctly on the respective platforms such that the application using this package can sample location data and process it in the background. Shortly after the start of this thesis, Google limited how the Android Location API handles background updates, which meant that an Android application must be in the foreground to track location such as is the case with Google Maps navigation where it is very explicit\footnote{\url{https://developer.android.com/training/location/background}}. Due to time constraints however, a good solution to this problem was not found.