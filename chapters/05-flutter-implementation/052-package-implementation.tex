\section{Package Implementation}
The package was implemented in Flutter according to the design in Chapter \ref{chapter:04} in which a series of components and the overall data model was outlined. This section will go through selected examples of source code as well as the general principles applied, to achieve the specified design, when implementing in Flutter and Dart. In the Dart programming language, fields, constructors and methods are declared private by using the underscore prefix, i.e. the field \verb|routineIndex| becomes \verb|_routineIndex|. This is used throughout the implementation to prevent the application programmer to access the inner parts of the package. The parts of the implementation related to storage of historical data can be found in Appendix \ref{appendix:source-code}. The offline algorithms for computing features were developed in Python with a large part being carried out by Jonas Busk\footnote{\url{https://www.researchgate.net/scientific-contributions/2129480702_Jonas_Busk}}. Source code for these algorithms can be found in Appendix \ref{appendix:python-demo}.

\subsection{Domain Model Implementation}
All the components specified in the Domain Model Chapter \ref{chapter:04} were implemented with their respective relations to each other. The only class with a public-facing constructor is \textit{LocationSample}, and by transitivity, also \textit{GeoPosition}. This is done to allow the user instantiate a \textit{LocationSample} with data from a given \textit{Location DTO}. The \textit{GeoPosition} class a field for the latitude and one for the longitude and a fundamental class used by the \textit{GeoSpatial} interface. The interface is a private abstract class which means it is only visible internally in the package library.

\begin{minted}{dart}
    abstract class _Geospatial {
      GeoPosition get geoPosition;
    }
\end{minted}

This interface allows other classes to promise the Dart compiler that it has a \textit{GeoPosition} field which allows it to be compared to other classes that implement the same interface. In Dart interfaces and abstract classes are the same thing, and the \textit{abstract class} keyword is used for implementing them. The \textit{GeoPosition} class even implements this interface since a \textit{GeoPosition} object itself has a \textit{GeoPosition}. This may seem superfluous but will come in handy when finding Stops (see Subsection \ref{subsection:finding-features}).

\begin{minted}{dart}
    class GeoPosition implements _Serializable, _Geospatial {
      double _latitude;
      double _longitude;
    
      GeoPosition(this._latitude, this._longitude);
    
      GeoPosition get geoPosition => this;
      double get latitude => _latitude;
      double get longitude => _longitude;
    }
\end{minted}


\subsection{Computing Features}
\label{subsection:finding-features}
Finding the location features Stops, Moves, and Places  were done according to the algorithms described in Chapter \ref{chapter:03}. 

\subsubsection*{Stops}
The Stop class has two constructors: A factory constructor which takes a set of LocationSamples from which the centroid of the set is computed, as well as the earliest timestamp, which will be the arrival time, and the latest timestamp which will be the departure time. After these attributes are found, the normal constructor is used.

\begin{minted}{dart}
    factory Stop._fromLocationSamples(List<LocationSample> locationSamples,
      {int placeId = -1}) {
      
      GeoPosition center = _computeCentroid(locationSamples);
      return Stop._(center, locationSamples.first.datetime,
        locationSamples.last.datetime, placeId: placeId);
    }
\end{minted}

The normal constructor uses a \textit{GeoPosition}, in addition to an arrival and departure time. A \textit{place ID }may also be specified at construction, but often it is not yet known at construction time hence it is optional.

\begin{minted}{dart}
    Stop._(this._geoPosition, this._arrival, 
        this._departure, {this.placeId = -1});
\end{minted}

The Stop algorithm takes a List of \textit{LocationSamples} as input and uses two while-loops, and two pointers (\textit{start} and \textit{end}) to delimit a subset of the input data. Every time the outer loop iterates, the \textit{start} pointer is moved past the end pointer, to skip already seen data. The inner loop is responsible for moving the \textit{end} pointer: With each iteration of the inner loop, the centroid of the current subset is computed. If the distance from this centroid to the latest added sample is within the given \verb|stopRadius| parameter, then the subset is expanded by incrementing the \textit{end} pointer, and the process is continued. Otherwise, the inner loop terminates and a Stop is created from the subset. The Stop is created without a \textit{place ID} since Places  have not yet been identified. Also, Stops with a duration shorter than the duration specified by the \verb|stopDuration| parameter are removed since they are noisy. This is an addition to the algorithms previously described and is mostly used due to the very high sampling frequency which was not accounted for the in original implementation by Cuttone et al. \cite{sparse-location-2014}.

\begin{minted}{dart}
    int start = 0;
    while (start < n) {
      int end = start + 1;
      List<LocationSample> subset = data.sublist(start, end);
      GeoPosition centroid = _computeCentroid(subset);
        
      while (end < n && 
        Distance.fromGeospatial(centroid, data[end]) <= stopRadius) {
        end += 1;
        subset = data.sublist(start, end);
        centroid = _computeCentroid(subset);
      }
    
      Stop s = Stop._fromLocationSamples(subset);
      stops.add(s);
    
      start = end;
    }
\end{minted}

The distance calculation is carried out using the \verb|GeoSpatial| interface previously mentioned. The distance function \verb|fromGeoSpatial| takes two objects which implement the interface and unpacks the latitude and longitude from these objects. The haversine distance can then be computed afterward.

\begin{minted}{dart}
    class Distance {
      static double fromGeospatial(_Geospatial a, _Geospatial b) {
        return fromList(
            [a.geoPosition._latitude, a.geoPosition._longitude],
            [b.geoPosition._latitude, b.geoPosition._longitude]);
      }
    
      static double fromList(List<double> p1, List<double> p2) {
        /// Haversine implementation
      }
    }
\end{minted}

\subsubsection*{Finding Moves}
The Move class has two constructors which are both private. Common for both constructors is that they take two Stops as arguments, with argument being either a path of \textit{LocationSamples} or a distance (a double). The factory constructor called \verb|_fromPath| calculates the distance of the path and then uses the normal constructor to create a Move. 
\begin{minted}{dart}
    factory Move._fromPath(Stop a, Stop b, List<LocationSample> path) {
      double d = _computePathDistance(path);
      return Move._(a, b, d);
    }
\end{minted}

\begin{minted}{dart}
    Move._(this._stopFrom, this._stopTo, this._distance);
\end{minted}
The normal constructor is used for de-serialization whereas the factory constructor is used to create a Move given two Stops and the path of samples between them.

The algorithm for finding Moves takes a List of \textit{LocationSamples} and the Stops found from the samples as input. The algorithm first checks if the set of Stops is empty, and if so returns an empty set of Moves. If, the set of Stops is not empty, then two 'fake' Stops are created and added to the set of Stops. These two additional stops are created from the first and last element in the set of Location Samples. For each Stop in the set of Stops, it is calculated which samples lie in between the current and next Stop. A Move is then created using the current Stop, the next Stop, and the path between. 

\begin{minted}{dart}
  Stop first = Stop._fromLocationSamples([data.first]);
  List<Stop> allStops = [first] + stops;

  if (data.first != data.last) {
    Stop last = Stop._fromLocationSamples([data.last]);
    allStops.add(last);
  }

  for (int i = 0; i < allStops.length - 1; i++) {
    Stop cur = allStops[i];
    Stop next = allStops[i + 1];
    List<LocationSample> samplesInBetween = data
        .where((d) =>
            cur.departure.leq(d.datetime) && d.datetime.leq(next.arrival))
        .toList();

    moves.add(Move._fromPath(cur, next, samplesInBetween));
  }
\end{minted}

The notion of fake Stops is an addition to the definition in Chapter \ref{chapter:03}, and are created to avoid edge cases where tracking was started while moving. In such a case no Moves will be created before the user is stationary for some time at least one Stop is found. This situation will likely not be very common, but was found during self-study and therefore deemed worthy of covering. The extra Stops are only used for finding moves and will not be used for finding Places.

\subsubsection*{Finding Places}
The Place class only has one normal constructor which takes an ID (an integer) and a List of Stops. 
\begin{minted}{dart}
Place._(this._id, this._stops);
\end{minted}

The Place algorithm takes a set of Stops for a given period, i.e. Stops over multiple days. The DBSCAN algorithm by Ester et al. \cite{density-based-1996} is used to find clusters in the Stops and label each Stop with a cluster-ID, this is the \textit{place ID} previously discussed. Once the labels are computed, the Stops are grouped by their \textit{place ID}. For each group, a Place object is created with the group label and all the Stops with that label. Lastly, the \verb|placeId| attribute for each Stop in the group is set to the group label.

\begin{minted}{dart}
  DBSCAN dbscan = DBSCAN(
      epsilon: placeRadius, minPoints: 1, 
      distanceMeasure: Distance.fromList);
  
  List<List<double>> stopCoordinates =
      stops.map((s) => ([s.geoPosition.latitude, 
        s.geoPosition.longitude])).toList();

  dbscan.run(stopCoordinates);

  Set<int> clusterLabels = dbscan.label.where((l) => (l != -1)).toSet();

  for (int label in clusterLabels) {
    List<int> indices =
        stops.asMap().keys.where((i) => (dbscan.label[i] == label)).toList();

    List<Stop> stopsForPlace = indices.map((i) => (stops[i])).toList();

    Place p = Place._(label, stopsForPlace);
    places.add(p);

    stopsForPlace.forEach((s) => s.placeId = p._id);
  }
\end{minted}


\subsubsection*{Hour Matrix Computation}
The Hour Matrix is an auxiliary feature used for internal computation and is therefore private. The class is implemented using a 2D double array as a field, representing the matrix of 24 rows, equal to the number of hours in a day, and columns equal to the Number of Places visited on the day. The construction of the Hour Matrix is done with a factory constructor that takes a List of Stops and the number of places visited. From this, the matrix is created and filled out. Each Stop can be converted into an array of doubles that tells which place and how much was visited.

\begin{figure}
    \centering
    \begin{minted}{dart}
    factory _HourMatrix.fromStops(List<Stop> stops, int numPlaces) {
      List<List<double>> matrix = new List.generate(
          HOURS_IN_A_DAY, (_) => new List<double>.filled(numPlaces, 0.0));
    
      for (int j = 0; j < numPlaces; j++) {
        List<Stop> stopsAtPlace = stops
            .where((s) => (s.placeId) == j).toList();
    
        for (Stop s in stopsAtPlace) {
          for (int i = 0; i < HOURS_IN_A_DAY; i++) {
            matrix[i][j] += s.hourSlots[i];
          }
        }
      }
      return _HourMatrix(matrix);
    }
    \end{minted}
    \caption{Construction of the Hour Matrix}
    \label{fig:hour-matrix-construction}
\end{figure}

Next, the \verb|routineMatrix()| factory constructor is discussed. This is a method for creating the RoutineroutineMatrix Matrix of the average day, given a list of other Hour Matrices. The method is quite simple since it uses two for loops to fill out an empty zero-matrix with the average value of each position indexed by \verb|i| and \verb|j|, for each matrix.

\begin{figure}
    \centering
    \begin{minted}{dart}
    factory _HourMatrix.routineMatrix(List<_HourMatrix> matrices) {
      int nDays = matrices.length;
      int nPlaces = matrices.first.matrix.first.length;
      List<List<double>> avg = zeroMatrix(HOURS_IN_A_DAY, nPlaces);
    
      for (_HourMatrix m in matrices) {
        for (int i = 0; i < HOURS_IN_A_DAY; i++) {
          for (int j = 0; j < nPlaces; j++) {
            avg[i][j] += m.matrix[i][j] / nDays;
          }
        }
      }
      return _HourMatrix(avg);
    }
    \end{minted}
    \caption{Computation of the Routine Matrix (i.e. average Hour Matrix)}
    \label{fig:hour-matrix-average}
\end{figure}

Lastly, the \verb|computeOverlap| method is discussed: This method computes the overlap similarity function discussed in Equation \eqref{eq:overlap-function}. Another Hour Matrix is provided as parameter referred to as \verb|other| and the current Hour Matrix is referred to as \verb|this| since the method is called on a specific object. 

The maximum possible overlap is computed as the minimum of the two matrix sums, since if one matrix is very sparse, then the overlap is severely limited. If either of the sums is zero then -1 is returned, due to either matrix being empty, which is valid. For computing total overlap a sum is used, and the matrix positions are iteration. For each position, the overlap for two scalars is computed and added to the total overlap. The overlap for two scalar values we defined in Equation \eqref{eq:overlap-function} as the minimum value of the two, given that both values are non-negative. 

\begin{figure}
    \centering
    \begin{minted}{dart}
    double computeOverlap(_HourMatrix other) {
      assert(other.matrix.length == HOURS_IN_A_DAY &&
          other.matrix.first.length == _matrix.first.length);
    
      double maxOverlap = min(this.sum, other.sum);
      if (maxOverlap == 0.0) return -1.0;
    
      double overlap = 0.0;
      for (int i = 0; i < HOURS_IN_A_DAY; i++) {
        for (int j = 0; j < _numberOfPlaces; j++) {
          if (this.matrix[i][j] > 0.0 && other.matrix[i][j] > 0.0) {
            overlap += min(this.matrix[i][j], other.matrix[i][j]);
          }
        }
      }
      return overlap / maxOverlap;
    }
    \end{minted}
    \caption{Construction of the Hour Matrix}
    \label{fig:hour-matrix-overlap}
\end{figure}


\subsubsection*{Home Stay Computation}
The derived features are computed according to their definitions in Chapter \ref{chapter:03}, using the lazy evaluation template outlined in Figure \ref{fig:lazy-evaluation}. The home stay feature is no exception. The algorithm for computing home stay uses the stops of today: First, the total time elapsed today is calculated using the departure timestamp of the last known Stop of today. Next, the Stops are used to identify the home place by constructing an Hour Matrix and then extracting the \verb|homePlaceId| from the Hour Matrix. Then, the total duration spent at the home place is calculated by summing the duration of the Stops which belong to the home place. The home stay is then calculated as the time at home divided by the total time elapsed.

\begin{figure}
    \centering
\begin{minted}{dart}
double _calculateHomeStay() {
  DateTime latestTime = _stops.last.departure;

  int totalTime = latestTime.millisecondsSinceEpoch -
      latestTime.midnight.millisecondsSinceEpoch;

  _HourMatrix hm = this.hourMatrix;
  if (hm.homePlaceId == -1) {
    return -1.0;
  }

  int homeTime = stops
      .where((s) => s.placeId == hm.homePlaceId)
      .map((s) => s.duration.inMilliseconds)
      .fold(0, (a, b) => a + b);

  return homeTime.toDouble() / totalTime.toDouble();
}
\end{minted}
    \caption{The method for computing the home stay feature}
    \label{fig:home-stay-code}
\end{figure}

\subsubsection*{Routine Index}
The routine index is the most difficult to compute by far. The method for computing this feature inside the Mobility Context class is however quite short, but this is due to all the matrix computations being done in the Hour Matrix class, i.e. the averaging and overlapping of matrices. The algorithm first checks if any contexts are provided if not then the routine index should be -1.0. Next, the Hour Matrices for each historic date is computed, and from these, the Routine Matrix is computed. Lastly, the routine index is found by computing the overlap between the two Hour Matrix of today, and the Routine Matrix, using the \verb|a.computeOverlap(b)| method of the Hour Matrix class.

\begin{figure}
    \centering
    \begin{minted}{dart}
    double _calculateRoutineIndex() {
      if (contexts == null) {
        return -1.0;
      } else if (contexts.isEmpty) {
        return -1.0;
      }
    
      List<_HourMatrix> matrices = contexts
          .where((c) => c.date.isBefore(this.date))
          .map((c) => c.hourMatrix)
          .toList();
    
      _HourMatrix routine = _HourMatrix.average(matrices);
    
      return this.hourMatrix.computeOverlap(routine);
    }
    \end{minted}
    \caption{The method for computing the routine index feature}
    \label{fig:routine-index-code}
\end{figure}

\subsection{Mobility Context}
A \textit{MobilityContext} object represents features for a given date. The class has a private constructor that takes a List of Stops and Moves from today, and a List of Places from the current period, i.e. the last 28 days including today. When the class is instantiated the date of today is automatically inferred, if not provided through the date parameter, which is an optional parameter. This parameter can be overridden in the case of unit testing for specific dates or if the programmer wishes to compute a Mobility Context for a date in the past.

\begin{minted}{dart}
MobilityContext._(this._stops, this._allPlaces, this._moves,
  {this.contexts, this.date}) {
  _timestamp = DateTime.now();
  date = date ?? _timestamp.midnight;
}
\end{minted}

The other optional parameter is a List of Mobility Contexts is used for computing the routine index - how this is achieved will be explained in Subsection \ref{subsection:context-generation}. The 'derived' features are implemented as doubles (except for Number of Places which is an integer) and are fields in the Mobility Context class.

\begin{figure}
    \centering
    \begin{minted}{dart}
    class MobilityContext {
        // Field
        double _routineIndex;
     
         // Getter
        double get routineIndex {
            return _routineIndex;
        }
    }
    \end{minted}
    \caption{The getter method for a feature field}
    \label{fig:feature-getter}
\end{figure}

\subsubsection*{Lazy Evaluation}
A \textit{getter} method for a given feature should reflect that the feature that is to be retrieved requires minimal computation (i.e. 'getting'), and therefore the feature computation should not take place in the getter method. However, there is a middle way, since the given feature needs to be computed only once to be evaluated, which allows us to keep the getter syntax. This middle way is \textit{lazy evaluation} described by Fowler \cite{fowler-PEEA} [p. 200], which applied to this example is the idea of postponing computation of a given field until the first time it is needed. After being computed, the field is stored and can be retrieved henceforth without any computational cost. In practice, this is done by letting the field be initialized to \textit{null}, and checking for \textit{null} in the getter method. If the value is \textit{null} then the feature is calculated and the field's value is updated after the computation and the getter can return the field's value. If the field is not \textit{null} then the feature has already been computed and can be returned immediately.

\begin{figure}
    \centering
    \begin{minted}{dart}
    class MobilityContext {
        // Field
        double _routineIndex;
        
        // Getter
        double get routineIndex {
            if (_routineIndex == null) {
                _routineIndex = _calculateRoutineIndex();
            }
            return _routineIndex;
        }
    }
    \end{minted}
    \caption{Lazy evaluation of a feature}
    \label{fig:lazy-evaluation}
\end{figure}


\subsection{Context Generation}
\label{subsection:context-generation}
The instantiation of Mobility Contexts is done through the Context Generator class, which is the interface between the programmer and the core of the package. All computation and storing and loading of data is done through this class. The class is a static class and thus does not have a mutable state which means that all methods of this class are also static and cannot rely on any internal, non-static values. The central part of the ContextGenerator class is the \verb|generate()| method, which is where a MobilityContext is computed. The method is asynchronous since it requires loading data from the file system before computation can take place. It does not require any parameters to call, but has two optional parameters: \verb|usePriorContexts| is a boolean option to compute the MobilityContext using prior contexts which is false by default. The other parameter is a date parameter, \verb|today|, which similar to the MobilityContext constructor allows the user to override today's date, which is automatically computed if not specified.

\begin{minted}{dart}
static Future<MobilityContext> generate(
      {bool usePriorContexts: false, DateTime today}) {...}
\end{minted}

First, the file system is queried by initializing the three different MobilitySerializers, i.e. one for Location Samples, another for Stops and a third one for Moves. Next, Location Samples are loaded and filtered; any samples with a date different from today are thrown away since they have already been used on a previous day and are no longer relevant to keep. After this the Stops and Moves are loaded from disk and filtered; any Stops and Moves that are either from today or older than 28 days are thrown away.

\begin{minted}{dart}
List<LocationSample> samplesToday = await sampleSerializer.load();
List<Stop> stopsHist = await stopSerializer.load();
List<Move> movesHist = await moveSerializer.load();

samplesToday = _filterSamples(samplesToday, today);
stopsHist = _stopsHistoric(stopsHist, today);
movesHist = _movesHistoric(movesHist, today);
\end{minted}

The reason for throwing away elements from today is that they need to be recomputing using all the recent Location Samples which can alter the old results. After recomputing today's Stops and Moves, the historical and recent Stops are merged to represent the whole period, and likewise for the Moves. Places are then computed using all the Stops of the period.

\begin{minted}{dart}
List<Stop> stopsToday = _findStops(samplesToday, today);
List<Move> movesToday = _findMoves(samplesToday, stopsToday);

List<Stop> stopsAll = stopsHist + stopsToday;
List<Move> movesAll = movesHist + movesToday;

List<Place> placesAll = _findPlaces(stopsAll);
\end{minted}

Next, the Stops and Moves for the period are stored to disk, but before they are stored, the flush method is used for the serializers in order to delete the old content permanently.
\begin{minted}{dart}
stopSerializer.flush();
moveSerializer.flush();
stopSerializer.save(stopsAll);
moveSerializer.save(movesAll);
\end{minted}

Lastly, if prior contexts are to be used then the historical dates are extracted from the historical stops, and for each date, the Stops and Moves are extracted and used to construct a Mobility Context, with each context being added to a List of prior contexts.

\begin{minted}{dart}
List<MobilityContext> priorContexts = [];

if (usePriorContexts) {
  Set<DateTime> dates = stopsHist.map((s) => s.arrival.midnight).toSet();
  for (DateTime date in dates) {
    List<Stop> stopsOnDate = _stopsForDate(stopsHist, date);
    List<Move> movesOnDate = _movesForDate(movesHist, date);
    MobilityContext mc =
        MobilityContext._(stopsOnDate, placesAll, movesOnDate, date: date);
    priorContexts.add(mc);
  }
}
\end{minted}

The method returns a MobilityContext object using the Stops and Moves of today and the Places for the period. Also, the date of today is chosen to be overridden and the computed contexts are also provided. If no contexts were computed, then \verb|priorContexts| will be an empty List.

\begin{minted}{dart}
return MobilityContext._(stopsToday, placesAll, movesToday,
        contexts: priorContexts, date: today);
\end{minted}

