\section{Thesis Overview}
The process of developing the Mobility Features Package is documented in the following chapters for which a brief overview is given below.\\

\textbf{Chapter 2: Related Work} will outline a set of relevant contributions related to behaviour and mental state tracking through mobile sensing. \\

\textbf{Chapter 3: Theoretical Background} will describe the mathematical theory behind computing the features. This includes algorithm design and definitions of all the features in mathematical terms. This section lays the ground work for the later implementation in any programming language.\\

\textbf{Chapter 4: Software Design} will discuss the design choices and principles used to model a software system capable of computing the mobility features. Design pertains to both how the internal system of the package works as well as the public facing API which is the interface the application programmer talks to when using the package.\\

\textbf{Chapter 5: Flutter Implementation} will describe how the domain model and the feature algorithms were implemented in the Dart language. How a user may use the final version of the package is also described.\\

\textbf{Chapter 6: Validation} will outline the field study that was conducted in which participants were tracked with a mobile application that used the Mobility Features Package to compute daily features based on their location data. Implementation of the mobile application is also discussed in this chapter.\\

\textbf{Chapter 7: Results and Discussion} will include a data analysis of the study in which the computed features are compared with the subjective answers given by participants. Additionally, further work is discussed including improvements to the package and future integration into other projects.\\

\textbf{Chapter 8: Conclusion} will report the major findings of the thesis and answer the research questions in detail.\\