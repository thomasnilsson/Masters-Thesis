\section{Background}
% Some authors provide a miniature literature review to give the reader enough background to understand the context of the research; a full review is usually deferred to the second chapter of the thesis. Other authors actually do the full review in Chapter 1, but that is rarer.

% Some questions you should be able to answer in general terms after reading this section are:
% This section will provide a context for the thesis in the form of a short introduction to the topic of deression and secondly a brief litterature review within mobile sensing and its applications.

% Depression and treatments
Major Depressive Disoder (MDD), commonly known as depression, is a serious illness that is currently treated with medication or through therapy. Dobson et al. \cite{randomized_trial_behavioural_activation} discusses how medication is the current treatment standard for severe cases of depression, with cognitive therapy, a form of psychotherapy, being the most widely investigated treatment for more moderate cases. Dobson et al. also discusses how not all patients wish to be medicated and cognitive therapy has not demonstrated efficacy across trials. Furthermore, traditional psychotherapy is resource-intensive to treat as it requires traditional methods that involves face-to-face consultations face to face consultations. Chartier and Provencher \cite{behavioural_activation_for_depression} describe the need to identify effective, evidence-based treatments that are time and cost-effective in an effort to increase the population's accessibility to treatments. Here, low-intensity interventions, such as guided self-help treatments, hold promise for the dissemination of evidence-based treatments. The treatment form of \textit{behavioral activation} is a component of cognitive behavioral therapy, is receiving increased attention and empirical support as a stand-alone psychotherapy method for treating depression.  Dobson et al. found that cognitive therapy was outperformed by behavioral activation which performed similarly to antidepressant medication. Furthermore, the findings were also that behavioral activation is less expensive and longer-lasting alternatives to medication in the treatment of depression. Gravenhorst et al. \cite{Gravenhorst2015} back up this claim stating how mobile phones, in particular in the area of mental disorders, can provide better treatment to more users with a lower cost. \\

% Mobile sensing
Mobile sensing involves the use of the phone's sensors to collect objective user data and may include predicting the user state from this data. Objectivity of the data stems from the data being collected from sensors in contrast to subjective user data such as questionnaires. Mobile sensing has become especially prevalent the last few years due to the rise of ubiquitous computing. Applications of mobile sensing include mental health monitoring also known as \textit{digital phenotyping} or \textit{user context generation}. Mobile sensing within healthcare is often abbreviated \textit{mHealth}. Within mental healthcare, mHealth opens the potential for practical and low-cost solutions for psychological interventions \cite{mobile-based-interventions}. In addition, mHealth also enables mental healthcare to reach populations that do not have access to traditional psychotherapy \cite{future-mental-health}, such as developing countries. Within the context of mHealth for behaviour and mental state, location data is commonly used. Rohani et al. \cite{rohani2018-correlations} found in a review of 46 studies that several of them showed consistent and statistically significant correlations between objective behavioral features collected via mobile and wearable devices and depressive mood symptoms. In addition they conclude that continuous- and everyday monitoring of behavioral metrics could be a promising supplementary objective measure for estimating depressive mood symptoms. As behavioural activation requires an understanding of the user's mental state and behavior it is an ideal use case for using objective mobility features. \\

