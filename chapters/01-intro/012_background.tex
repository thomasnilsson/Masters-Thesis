\section{Background}
% Some authors provide a miniature literature review to give the reader enough background to understand the context of the research; a full review is usually deferred to the second chapter of the thesis. Other authors do the full review in Chapter 1, but that is rarer.

% Some questions you should be able to answer in general terms after reading this section are:
% This section will provide a context for the thesis in the form of a short introduction to the topic of depression and secondly a brief literature review within mobile sensing and its applications.

% Depression and treatments
Major Depressive Disorder (MDD), commonly known as depression, is a serious illness that is currently treated with medication or through therapy. Dobson et al. \cite{randomized_trial_behavioural_activation} discusses how medication is the current treatment standard for severe cases of depression, with cognitive therapy, a form of psychotherapy, being the most widely investigated treatment for more moderate cases. Traditional psychotherapy is resource-intensive since it relies on traditional methods that involve face-to-face consultations. Chartier and Provencher \cite{behavioural_activation_for_depression} describe the need to identify effective, evidence-based treatments that are time and cost-effective to increase the population's accessibility to treatments. The treatment form of \textit{behavioral activation}, a component of cognitive-behavioral therapy, is receiving increased attention and empirical support as a stand-alone psychotherapy method for treating depression. Dobson et al. found that behavioral activation performed similarly to antidepressant medication. Furthermore, the findings were also that behavioral activation is less expensive and longer-lasting alternatives to medication in the treatment of depression.  \\

% Mobile sensing
Mobile sensing involves the use of the phone's sensors to collect objective user data and may include predicting the user state from this data. The objectivity of the data stems from the data being collected from sensors in contrast to subjective user data such as questionnaires. Mobile sensing has become especially prevalent in the last few years due to the rise of ubiquitous computing. Applications of mobile sensing include mental health monitoring also known as \textit{digital phenotyping} or \textit{user context generation}. Within mental healthcare, mHealth opens the potential for practical and low-cost solutions for psychological interventions \cite{mobile-based-interventions}. MHealth also enables mental healthcare to reach populations that do not have access to traditional psychotherapy \cite{future-mental-health}, such as developing countries. Gravenhorst et al. \cite{Gravenhorst2015} describes how mobile phones, in particular in the area of mental disorders, can result in a better treatment that can reach more users, at a lower cost compared to traditional methods.\\ 

Within the context of mHealth for behavior and mental state, location data is commonly used. Rohani et al. \cite{rohani2018-correlations} found in a review of 46 studies that several of them showed consistent and statistically significant correlations between objective behavioral features collected via mobile and wearable devices and depressive mood symptoms. In addition, they conclude that continuous- and everyday monitoring of behavioral metrics could be a promising supplementary objective measure for estimating depressive mood symptoms. As behavioral activation requires an understanding of the user's mental state and behavior it is an ideal application for using objective mobility features. \\


