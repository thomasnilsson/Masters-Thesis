\section{Background}
% Some authors provide a miniature literature review to give the reader enough background to understand the context of the research; a full review is usually deferred to the second chapter of the thesis. Other authors actually do the full review in Chapter 1, but that is rarer.

% Some questions you should be able to answer in general terms after reading this section are:
% This section will provide a context for the thesis in the form of a short introduction to the topic of deression and secondly a brief litterature review within mobile sensing and its applications.

Major Depressive Disoder (MDD), commonly known as depression, is a serious illness that is resource-intensive to treat using traditional methods such as face to face consultations. This costs society a lot of money, and it is, therefore, worthwhile to explore treatments that can improve the quality of life of these patients. 

\subsection*{Treatment with Digital Phenotyping}
Among many treatments, Behavioural Activation is very well supported by mobile sensing technologies which have become a growing field due to the rise of ubiquitous computing. Currently, the diagnosis in bipolar disorder, also known as manic depression, relies on manual patient information and clinical evaluations and judgments with a lack of objective testing. Some of the essential clinical features of individuals suffering from bipolar disorder are changes to their daily behavior \cite{objective_smartphone_data_as_diagnostic_marker}. These changes in behaviour can be captured by the user's \textit{mobility context}, and can thus provide very helpful in customizing a Behavioural Activation treatment. A mobility context can be generated by user rules and triggered by events from semi- or full- automated algorithms. Here, it will be investigated how the context can be generated in a fully-automatic, real-time fashion.

% In general, who has looked at this area before?
\subsection*{Previous Work}
Clustering location data into places has been done previously in a more general context by Ashbrook and Starner \cite{learning_significant_locations}which used data tolearn significant locations and predicting where the user will move next. Cuttone et al. \cite{sparse-location-2014} have contributed to the field by finding that even low accuracy sampling can be used to identify significant user locations. Recently, location data has also been used within the context of depression by Saeb et al. \cite{Saeb2015} and Canzian et al. \cite{Canzian2015}. These two contributions define a list of mobility features to track changes in user behaviour. In addition Saeb et al. \cite{Saeb2015} also showed that certain mobility features correlate strongly with scoring highly on the Patient Health Questionnaire (PHQ-9) (see Appendix \ref{appendix:questionnaires}), which indicates an individual is depressed. In regards to Behavioural Activation, Rohani \cite{mubs-rohani, moribus} developed a recommender system for treating depression while working on his PhD at CACHET. The current system relies on manual user input, and an ideal improvement would be to incorporate automatically generated mobility features recommendation-algorithm. Lastly, the work done at CACHET on the CARP Mobile Sensing Framework\footnote{\url{https://github.com/cph-cachet/carp.sensing-flutter/wiki}} within mobile sensing provides an ecosystem in which the resulting package would fit in.