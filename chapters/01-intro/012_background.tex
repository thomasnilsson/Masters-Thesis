\section{Background}
% Some authors provide a miniature literature review to give the reader enough background to understand the context of the research; a full review is usually deferred to the second chapter of the thesis. Other authors actually do the full review in Chapter 1, but that is rarer.

% Some questions you should be able to answer in general terms after reading this section are:
% This section will provide a context for the thesis in the form of a short introduction to the topic of deression and secondly a brief litterature review within mobile sensing and its applications.


Major Depressive Disoder (MDD), commonly known as depression, is a serious illness that is resource-intensive to treat using traditional methods such as face to face consultations. This costs society a lot of money, and it is, therefore, worthwhile to explore treatments that can improve the quality of life of these patients. Among many treatments, \textit{Behavioural Activation} is very well supported by mobile sensing technologies \cite{behavioural_activation_for_depression} which have become a growing field due to the rise of ubiquitous computing. Currently, the diagnosis in bipolar disorder, also known as manic depression, relies on manual patient information and clinical evaluations and judgments with a lack of objective testing. Some of the essential clinical features of individuals suffering from bipolar disorder are changes to their daily behavior \cite{objective_smartphone_data_as_diagnostic_marker}. Mobile sensing is the field of using the phone's sensors to collect data user data and can include predicting the user state from this data. Some applications of mobile sensing include mental health monitoring, for which different terms are used such as "digital phenotyping" and "context generation". Mobile sensing for mental healthcare has the potential of providing a practical and low-cost approach to deliver psychological interventions for the prevention of mental health disorders \cite{mobile-based-interventions} as well as bringing mental healthcare to populations that do not have access to traditional psychotherapy \cite{future-mental-health}. Within the context of mobile sensing for behaviour and mental state, location data is commonly used. Namely the work by Saeb et al. \cite{Saeb2015} showed that location data can be used to derive certain \textit{mobility features} that are strongly connected with the PHQ-9 score for quantifying depression (see Appendix \ref{appendix:questionnaires}). Therefore, it would seem Behavioural Activation is an ideal use case for using mobility features to infer the user's mental state and behaviour. 