\section{Goals and Methods}
% While the above section details the problems, your job is to then translate those problems into research goals. Each goal should briefly indicate how you are going to solve the problem i.e., the method you will use to solve it. Note that some authors sometimes combine problem statements/goals/methods into a single section, while others separate it. Goals should be operational, i.e., if you later claim to achieve your goal, you should be able to match your solution against the goal statement.

% I cannot overstate how important it is to have clear goals. Most examiners highlight these goals, and after reading the thesis they then go back to see if you have actually accomplished your goals. If you have not, then you have a big problem with your thesis. Even worse are theses where problems and goals are not clearly stated, for it means people are trying to evaluate your solutions in a vacuum.

The research goals closely match the research questions and will explain how the different sub-questions are answered, and which methods are used to do so.

\subsubsection*{Goal 1: Implement the Off-line Mobility Feature Algorithms}
Saeb et al. \cite{Saeb2015} and Canzian et al. \cite{Canzian2015} describes a list of mobility features, some of which they prove to strongly correlate with depressive behavior. These features need to be implemented in their most simple form, i.e. off-line, where all the location data for a given day is readily available. This will be carried out in Python with a synthetic dataset and later in Dart. This is done in order to demonstrate the quality of the algorithms. A pre-processing algorithm inspired by Cuttone et al. \cite{sparse-location-2014} for finding certain features was used.

\subsubsection*{Goal 2: Adapt Algorithms to Work in Real-time}
The features described above will be implemented such that they can be evaluated on a partial dataset, i.e. an incomplete day's data. Many of the features will have minor changes applied to their calculation and a larger effort is needed to ensure real-time capabilities. This adaptation will be carried out in Python.

\subsubsection*{Goal 3: Implement Real-time Mobility Features on the Smartphone}
An procedure that runs the mobility feature algorithms on a smart-phone has to be written which involves a couple of steps. Location data has to be collected and stored on the smartphone, and then it has to be demonstrated that features can be computed whenever they need to be. However, storing raw data every day would end up taking up too much disk space and is therefore not an ideal solution when working with a smartphone. This means the solution has to be able to compute these history-dependent features without relying on storing raw data of each day. This implementation will be carried out in the Dart programming language.

\subsubsection*{Goal 4: Design and Release Flutter Software Package}
The software package should be designed to perform feature computation in a way that is easy to use for an application programmer. In addition this package should have proper documentation and be released on the Dart package manager \footnote{\url{www.pub.dev}}.

\subsubsection*{Goal 5: Develop a Demo Mobile Application}
To demonstrate how the software package containing the algorithm will be used in practice, a Flutter demo app will be developed in parallel with the software package. By developing an actual application and seeing the algorithms in use, a which is easy to use from an application programmer's perspective will emerge.

\subsubsection*{Goal 6: Conduct a Field Study}
To test the quality of the mobility features, a small scale study of 5-10 participants is planned to run for at least 2 weeks. The participants will use the demo application in which they will have to fill out a small diary in order to evaluate the features computed by the algorithms. Participants should be reminded daily to fill out the diary such that as much user data is collected as possible.

