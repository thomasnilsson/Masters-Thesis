\section{Results}
The Mobility Features Package was implemented in the Flutter framework and supports on-device real-time mobility feature computation. From the work of Saeb et al. \cite{Saeb2015} and Canzian et al. \cite{Canzian2015} the features \textit{Home Stay, Number of Places, Distance Travelled, Location Variance, Entropy, Normalized Entropy,} and \textit{Routine Index} were chosen to be included in the package. A set of additional location features (stops, places and moves) were included as well, inspired by the work of Cuttone et al. \cite{sparse-location-2014} and Canzian et al. \cite{Canzian2015}. Features are computed in real-time using historical stops and moves stored on the phone which lowers the storage requirements greatly compared to using raw location data for feature computation. The Mobility Features Package makes it possible for the application programmer to compute mobility features with just 3 lines of code. This comes as the result of the API concealing the feature implementation details from the programmer to very high degree. Location tracking is not automatically carried out by the package, this must instead be done by an application programmer when writing his/her mobile application. This allows for easier maintenance and enables the package to be used among other packages dependent on location tracking, without causing dependency issues. In addition the package was tested through a field study with 10 participants for which an application was developed in Flutter that used the package. The application collected the participants' location several thousand times per day which resulted in a dataset of 240 MB of raw location data in total. Participants filled out a daily questionnaire pertaining to the features, these answers were then compared to the computed features. In comparing the answers to the computed features, promising results were achieved for the participants that were diligent in tracking their location and providing answers. 