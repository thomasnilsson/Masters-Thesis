\section{Results}
The Mobility Features Package was implemented in Flutter which supports real-time mobility feature computation. From the work of Saeb et al. \cite{Saeb2015} and Canzian et al. \cite{Canzian2015} the features \textit{Home Stay, Number of Places, Distance Travelled, Location Variance, Entropy, Normalized Entropy, Routine Index} were chosen to be included in the package. A set of additional \textit{intermediate features}, Stops, Places and Moves were included as well. These intermediate features were developed by Jonas Busk, inspired by the work of Cuttone et al. \cite{sparse-location-2014}. Features are computed in real-time using historical intermediate features stored on the phone. By storing intermediate features rather than raw location data the storage requirements are reduced greatly. The Mobility Features Package makes it possible for the application programmer to compute mobility features with just 3 lines of code, which is due to the API concealing the feature implementation details from the programmer. Location tracking is not automatically carried out by the package, this must instead be done through the application using the package. This allows the package to be used among other packages dependent on location data, without causing dependency issues. In addition the package was validated through a field study with 10 participants. For this study, an application was developed in Flutter which used the package. The application collected the participants' location several thousand times per day which resulted in a total of 240 MB of raw location data. Participants filled out a daily questionnaire pertaining to the features, these answers were then compared to the computed features. In comparing the answers to the computed features, promising results were achieved for the participants which were diligent in tracking their location and providing answers. 