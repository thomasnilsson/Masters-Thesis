\section{Research Question}
% This section provides a very concise statement of your hypothesis/thesis/problems. The hypothesis or thesis is the highest-level problem or goal you are going to address. The specific list of problems --- usually a handful, although sub-problems are sometimes given --- are things that need to be solved if you are going to satisfy your hypothesis/thesis. Problems should be stated unambiguously. The importance of the problem should be mentioned if it hasn't already been done so in the prior sections. Of course, the problem must be worthy of a thesis.

Existing contributions within the field of generating mobility features and context are cumbersome, if even possible, to reproduce with regards to their source code. Furthermore, the algorithms that exist in the literature do not necessarily lend themselves to the real-time computation of features. How these two things can be achieved will be investigated in this thesis and are addressed in the following research questions:\\

\textbf{Question 1:} \textit{Which mobility features are relevant to include in a software package?}\\

\textbf{Question 2:} \textit{How can these features be computed in real-time, on a smartphone device?}\\

\textbf{Question 3:} \textit{How does the API design of such a software package look like?}\\
