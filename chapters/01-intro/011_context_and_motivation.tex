\section{Context and Motivation}
% Almost every Chapter 1 begins with a section that sets the scene and that motivates the problem being studied. It describes some domain and indicates a problem in general terms.

% Some questions you should be able to answer after reading the motivation section are:

% What is the general area being addressed?
% USE AN EXAMPLE, too abstract
% Det har vist sig, at man ved at opsamle locatio ndata, kan man sige meget om adf'rd og mentale tilstand. Fx der var et studie som viste (saeb) en høj korrelation mellem features og spørgeskemaer

Research has shown that changes in a user's location can reveal a lot about their behaviour and mental state \cite{Saeb2015, Canzian2015, palmius2017, }. A set of daily metrics called \textit{mobility features} can be derived from collected location data which correlate highly with the users mental state. This coherence between mobility features derived objectively from a user's phone and their mental state could be a vital step to support psychotherapy components in the clinic.\\

Numerous contributions exist within the field of using smartphone data for predicting the various states of the user. It is however not common practice for researchers to publish their source code such that it may be used by other researchers in the future. In addition, most research in the field of mobility data (see \cite{Saeb2015, saeb2016, Canzian2015, extraction-of-behavioural-features}) seems to develop for the Android platform exclusively since this platform is more prevalent in most countries\footnote{\url{https://gs.statcounter.com/os-market-share/mobile/worldwide}}. This means even if the code was released, researchers interested in using the iOS platform would have to rewrite the source code more or less from the ground up. Furthermore, source code tends to get deprecated over time, meaning maintenance by the researcher is required. \\

Furthermore, it is common practice to gather all of the data first and then perform feature computation and data analysis together afterwards \cite{Saeb2015, saeb2016, sparse-location-2014, extraction-of-behavioural-features}. This means that features are not computed when the mobile application is used 'in the field' which is when features can be used for interventions. The algorithms used in the literature therefore have not been considered for real-time computation, which has implications on algorithm design and resource constraints. Some features rely on historical data being available which also complicates real-time computation, since historical data will need to be stored and computed on a resource constrained device. As a consequence, researchers within mobile-health using this package will be able to move their focus from away from software engineering to focus on data analysis and study design.\\

This thesis describes the design and implementation of the \textit{Mobility Features Package}, a software package capable of computing mobility features in a smart-phone application. The research problem is focused on the software engineering aspect of developing such a package and will address the lack of re-usable source code in research and lack of support for on-device, real-time feature computation. By real-time we refer to on-device computation at any time of the day, that is sufficiently quick to not hinder the application.