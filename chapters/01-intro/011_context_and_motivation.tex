\section{Context and Motivation}
% Almost every Chapter 1 begins with a section that sets the scene and that motivates the problem being studied. It describes some domain and indicates a problem in general terms.

% Some questions you should be able to answer after reading the motivation section are:

% What is the general area being addressed?
% USE AN EXAMPLE, too abstract
% Det har vist sig, at man ved at opsamle locatio ndata, kan man sige meget om adf'rd og mentale tilstand. Fx der var et studie som viste (saeb) en høj korrelation mellem features og spørgeskemaer

It has been shown by Saeb et al. and Canzian et al. \cite{Saeb2015, Canzian2015} that a user's location data can reveal a lot about their change in behaviour and mental state. This is done by computing certain daily \textit{mobility features} from the location data over a period of time. In the context of mental health research, Saeb et. al \cite{Saeb2015} found that certain features correlate highly with the mental state reported by patients through medical questionnaires which indicate the features may be used as a component in psychotherapy. This thesis will describe the design and implementation of the \textit{Mobility Features Package}, a software package capable of computing mobility features in a smart-phone application. The research problem addressed in this project is focused on the software engineering aspect of developing the package and will focus on the lack of re-usable source code in research and lack of support for on-device, real-time feature computation. By real-time we refer to on-device computation at any time of the day, that is sufficiently quick to not hinder the application.\\

Regarding source code, numerous contributions exist within the field of using smartphone data for predicting the various states of the user. However, it is not common practice for researchers to publish their source code such that it may be used by other researchers in the future. Also, most research in the field of mobility data (see \cite{Saeb2015, saeb2016, Canzian2015, extraction-of-behavioural-features}) seems to develop for the Android platform exclusively since this platform is more prevalent in most countries. This means even if the code was released, researchers interested in using the iOS platform would have to rewrite the source code more or less from the ground up. Also, source code tends to get deprecated over time, meaning maintenance by the researcher is required. These problems are contributing to the lack of source code available.\\

Regarding the lack of real-time support, it is common practice to gather all of the data first and then perform feature computation and data analysis together afterwards. This means that features are not computed when the mobile application is used 'in the field' which is when features can be used for interventions. Some features rely on historical data being available which also complicates their real-time computation, since historical data will need to be stored and computed on a resource constrained device. Researchers within the field of mental health illnesses from a health-tech background will benefit greatly from using a Flutter package with a high level of abstraction, in which the researcher's focus is moved from software engineering to data analysis and study design. 