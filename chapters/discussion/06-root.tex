\chapter{Discussion}
\section{Results}

\section{Future Work}
Integrating this into apps people already\\
Give concrete recommendations, ex Rohani MUBS\\

\section{User Prompting in Real-Time}
When should the user be given recommendations\\
Trigger can be time interval or threshold for a feature, i.e. if homestay greater than 0.5 prompt the user to leave the house\\
Recommendations from depression database (Rohani, 300 items, not part of this thesis)\\

Calculate routine index from week to week rather than day to day. Currently we assume that all days ideally look identical, which for most people will not be the case. For many people the weekends differ quite a lot from their every day, and naturally the routine index will therefore be lower during the weekend, and weekend days will tend to 'wash out' the routine index calculation. 

\section{Calculations}
\subsection{Home Stay}
It was found that the total time spent at stops was sometimes differing from the  total amount of time in a day, i.e. 24 hours, and was often quite a bit lower. This meant the proposed home stay would be quite high. There were three different approached considered, which where to sum the total amount of time spent at stops on the day, consider the total to be the difference between the arrival at the first stop and departure from the last stop, or lastly to simply let it be defined as the amount of time of day, i.e. the home stay calculated at 13:30 would consider the total to be 13.5 hours.  