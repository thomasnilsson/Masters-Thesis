\section{Digital phenotyping: a global tool for psychiatry}
The paper \cite{digital_phenotyping} deals with the topic of collecting and aggregating user data into a so-called digital phenotype. It is predicted that by 2050, the biggest impact in psychiatry and mental health will have been the revolution in technology- and information science. Smartphones have become ubiquitous in the past decade and there are over three billion smartphones with a data plan worldwide, each of which have computing power which surpasses supercomputers of the 1990s. In areas around the world without easy access to clean water, ownership of a smartphone and by proxy, rapid access to information has become a symbol of modernity. 

\subsection{Phenotyping and Psychiatry}
In the realm of psychiatry, a current data collection problem is the dependence on self-reporting of sleep, appetite and emotional state, even though it is recognized that depression will impair peoples ability to remain objective in assessing their own behaviour and thus data is prone to be faulty \cite{digital_phenotyping}. Another current problem is how it how people suffering from mental illness tend to not seek help before it is too late. Depressive relapses are therefore also often reported with considerable delay for patients currently in treatment. 

The smartphone offers an objective form of mental-state measurement which is referred to was \textit{Digital Phenotyping}, which uses built-in sensors such as geo-location, accelerometer, and human-computer interactions (HCI) to infer the state of the patient. This makes it possible to assess people by using data in a real-time fashion, rather than in retrospect as is currently is done. Digital phenotyping could in theory fill the role of a smoke detector which provides early signs of relapse and recovery, without replacing the face-to-face consultations entirely. In addition, this also allows researchers to track patients in their own environment, rather than in a clinical environment.

\subsection{The Ethical Dimension}
However, when does measurement become surveillance? Is phenotyping by using data such as geo-location too invasive? A series of ethical issues have to be addressed before digital phenotyping can realistically hope to be employed as a tool for population health. Although, some of these issues have technical solutions, such as with tracking human-computer-interactions which is mostly related to \textit{how} the user interacts with a device, i.e. the pacing with which is typed, scrolled or how long phone calls last - rather than the content of what is provided to the device i.e. what is typed, spoken during a phone call, or the websites visited. 
