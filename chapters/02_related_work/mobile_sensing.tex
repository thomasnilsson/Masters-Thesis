% Detecting Bipolar Depression From Geographic Location Data
Various contributions The aim was to identify depressive episodes of subjects using geo-location recorded from smartphones, with respect to a broader community study of Bipolar Disorder (BPD). Data was collected over 3 months, where participants also reported their depressive symptoms using a weekly questionnaire. The findings were that there is a strong link between geographic movements and depression in BPD. 

% Personalised modelling of geographic movements in depression
Depression is a very serious and very prevalent among adults. Monitoring patients over a period of time can improve the life of the patients, and lessen the burden on the healthcare system in moderate cases of depression. An avenue which is emerging is the use of geo-location tracking to monitor patients, and to assess the mental health of the patients. A study was conducted with 130 participants, which included patients with BPD and Borderline. Features were extracted from the raw GPS data and a clustering method for stationary places is described. Finding are that god location is a marker of depression but also that handling data appropriately is essential for maximizing accuracy.

% Digital biomarkers from geolocation data in bipolar disorder and schizophrenia: a systematic review
Exploring what extent gps data has been used to study serious mental illness such as BPD and schizophrenia. Monitoring with a serious mental illness is largely done face-to-face, whereas gps-based monitoring can be done remotely and in a more continuous fashion. A meta study was reviewed, and the vast majority of papers found an association between GPS-derived markers and mood which was stronger than other signals such as accelerometer data. 