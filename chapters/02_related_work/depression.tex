\section{Depression in Context}
Major depression MDD \cite{major-depressive-disorder} is a commonly occurring, seriously impairing, and often recurrent mental disorder. The World Health Organization ranks major depressive disorder as the 4th leading cause of disability worldwide \cite{murray1996} and projects that by 2020 it will be the second leading cause due to currently unexplained increasing prevalence in recent cohorts. Projections for 2030 using WHO’s estimates of mortality and  burden of disease for 2002 \cite{projections_of_burden_of_disease}. Previously a model was made in 1990 but it did not account for HIV/AIDS, and in addition the numbers were somewhat outdated. The  three leading causes of burden of disease in 2030 are projected to include HIV/AIDS, (unipolar) depressive disorders and heart disease. 

The illness is characterized by depressed mood, low interest in doing anything, impaired cognitive function, and low quality of sleep, and changes in appetite \cite{mdd}. The illness is twice as prevalent in women, and on average affects one in six adults during their lifetime. The causes for MDD are many, among them is a heritability of 35\% and many environmental factors such as problems during childhood, however no perfect model for explaining the disease fully currently exists. Early-onset MDD is found to lead to difficulties later in life, including low educational attainment, teen pregnancy, divorce, and unstable employment \cite{costs_of_depression}. Furthermore, MDD is also a predictor for elevated risk of onset, persistence, and severity of a wide range of physical disorders and an increased rate of early mortality as well as suicide. Although research shows that treatment can reverse many adverse effects, only a minority of people with MDD receive treatment and many who do receive treatment receive extensive-, high-quality treatment. The current treatment is to manage the illness through psychotherapy as well as medication. Given the prevalence of depression, there is a need to find effective, evidence-based, cost-effective treatments for the broader public. 

\section{Behavioral Activation Treatment}
Activity Scheduling is a behavioral treatment for patients suffering from depression \cite{comparative_effectiveness_psycho_treatments}. in which patients monitor their daily activities as well as their mood. The goal is for patients to learn how they can increase the number of pleasant activities and positive interactions with their environment, using their own data. Low-intensity treatments such as guided self-help look promising - especially the field of  Behavioural Activation (BA), which is a component of Cognitive-Behavioural Therapy, which has gotten an increased amount of attention. The findings were that BA can be a viable option as low intensity, guided self-help treatment for mild to moderate cases of depression \cite{behavioural_activation_for_depression}. One meta-study finds that this approach shows promise even when there is a reduced number of face to face therapy sessions or even when done remotely \cite{comparative_effectiveness_psycho_treatments}. A second meta-study also concludes that it is an attractive option for treating depression, due to its relatively uncomplicated nature as well as being resource-efficient \cite{behavioural_activation_meta_analysis}.
%
Low-intensity treatments such as guided self-help looks promising - especially the field of  Behavioural Activation, which is a component of Cognitive-Behavioural Therapy, which has gotten an increased amount of attention. The findings were \cite{behavioural_activation_for_depression} that BA can be a viable option as a low intensity, guided self-help treatment for mild to moderate cases of depression. 
%
A broad selection of psychological treatments have been developed and are being used in primary care, including cognitive behavioural approaches. A meta-study \cite{comparative_effectiveness_psycho_treatments} finds that this approach shows promise even  when there is a reduced number of face to face therapy sessions or even remotely.