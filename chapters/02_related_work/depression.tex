\section{Depression in Context}
%\section{Major Depressive Disorder}
Major depressive disorder (MDD) is a very serious illness characterized by depressed mood, low interest in doing anything, impaired cognitive function and low quality of sleep and changes in appetite \cite{major-depressive-disorder}. The illness is twice as prevalent in women, and on average affects one in six adults during their lifetime. The causes for MDD are many, among them is a heritability of 35\%, i.e. what would be regarded as ‘nature’. The environmental factors, i.e. ‘nurture’, include a variety of abuses during childhood, however no perfect model or mechanism currently exists, which can explain the disease fully. MDD is associated with changes to the hippo-campus, responsible for the learning and memory of the brain as well as derived effects which affect the immune system. The current treatment is to manage the illness through psychotherapy as well as medication. In severe cases of MDD treatment may include electroconvulsive therapy, where the patient is given electroshock in order to affect the brain chemistry of the patient. ECT is believed to reverse symptoms of a range of mental conditions.

% \section{The costs of depression}
Early-onset MDD is found to lead to difficulties later in life, including low educational attainment, teen pregnancy, divorce, and unstable employment. Furthermore MDD is also a predictor for elevated risk of onset, persistence and severity of a wide range of physical disorders and increased rate of early mortality as well as suicide. Although trtials show that treatment can reverse many adverse effects, only a minority of people with MDD receive treatment, and many who do receive treatment receive an extensive-,  high quality treatment.
Major depression is a commonly occurring, seriously impairing, and often recurrent mental disorder.1, 2 The World Health Organization ranks major depressive disorder (MDD) as the 4th leading cause of disability worldwide3 and projects that by 2020 it will be the second leading cause due to currently unexplained increasing prevalence in recent cohorts.

% \section{Projections of global mortality and burden of disease from 2002 to 2030}
Projections for 2030 using WHO’s estimates of mortality and  burden of disease for 2002. Previously a model was made in 1990 but it did not account for HIV/AIDS, and in addition the numbers were somewhat outdated. The  three leading causes of burden of disease in 2030 are projected to include HIV/AIDS, (unipolar) depressive disorders and heart disease. 

% \section{Behavioral activation treatments of depression: a meta-analysis.}
Activity Scheduling is a behavioural treatment for patients suffering from depression. The basic idea is for patients to monitor their daily activities as well as their mood, and the goal is for patients to learn from their own data, how they can increase the number of pleasant activities and positive interactions with their environment. A meta-analysis of 16 studies of 780 subjects was conducted, with the findings that activity scheduling is an effective method for treating patients suffering from depression. The conclusion is therefore that it is an attractive option for treating depression given its relatively uncomplicated nature as well as being resource-efficient. 

% \section{Comparative effectiveness of psychological treatments for depressive disorders in primary care: network meta-analysis}
A broad selection of psychological treatments have been developed and are being used in primary care, including cognitive behavioural approaches. A meta-study finds that this approach shows promise even  when there is a reduced number of face to face therapy sessions or even remotely.

% \section{Behavioural activation for depression: efficacy, effectiveness and dissemination.}
Given the prevalence of depression, there is a need to find effective, evidence-based, cost-effective treatments for the broader public. Low-intensity treatments such as guided self-help looks promising - especially the field of  Behavioural Activation, which is a component of Cognitive-Behavioural Therapy, which has gotten an increased amount of attention. The findings were that BA can be a viable option as a low intensity, guided self-help treatment for mild to moderate cases of depression.
