\section{Rohani: The MORIBUS \& MUBS Systems}
The prevalence of depression has led to the development of many mobile health-tech solutions and applications for monitoring diseases related to depression. However few of these applications have focused on the treatment of depression. There is clinical evidence showing that depressive symptoms can be alleviated through Behavioural Activation which is a behavior-focused treatment method for treating depression. Rohani is a postdoc at CACHET and has published and developed two systems during his Ph.D. within the realm of Behavioural Activation which is the MORIBUS- and MUBS systems.

\subsection{The MORIBUS System}
The \textit{MORIBUS} system \cite{moribus} by Rohani aims to make it easy for depressed patients to remember and register their behavior in order to increase their every-day positive outcomes. The system comes with a catalog of 54 pre-made activities such as \textit{Get ready in the morning} and \textit{Eat breakfast} and has support for patients to add their own activities by manually inputting text and a rating. When an activity is completed, the patient registers this in the app along with a short reflection. The application is able to provide the user with statistics and summaries of activity frequency and reflections.

\subsection{MUBS}
The \textit{MORIBUS} system later evolved into the \textit{MUBS} system which was rewritten in Flutter \cite{mubs-rohani}. The application uses a larger catalog of pleasant activities and gives recommended activities to perform, personalized to each user. The catalog is a database of 384 common enjoyable activities where each activity has been manually labeled by researchers with an \textit{activity level}, as well as a \textit{category}. The recommender model uses a Naive Bayes model with a set of features representing each feature and a corresponding label representing whether it is positive or negative for that individual user. The model is designed to recommend already-known activities as well as novel activities to the user which is in the spirit of behavioral activation. Future work of the system includes adding additional features such as \textit{Location}, \textit{Local Weather} and \textit{Ambient Noise} to provide an even more detailed context to the system.

\subsection{Improving the Systems}
Both systems rely on manual input from the user in order to learn which activities to recommend in the future. This can pose a problem, especially when running longitudinal studies since participant retention may suffer if the input process is too involved for the user. These systems by Rohani, therefore, provide examples of use cases for the automated feature generation algorithm such as the \textit{Mobility Features Package} which is easy to integrate into an existing Flutter application.