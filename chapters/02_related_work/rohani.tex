\section{Rohani: The MORIBUS \& MUBS Systems}
The prevalence of depression has led to the development of many mobile health-tech solutions and applications for monitoring diseases related to depression. However few of these application have focused on the treatment of depression. There is clinical evidence showing that depressive symptoms can be alleviated through Behavioural Activation which is a behaviour-focused treatment method for treating depression. D. Rohani is a postdoc at CACHET and has published and developed two systems during his PhD within the realm of Behavioural Activation which are the MORIBUS- and MUBS systems.

\subsection{The MORIBUS System}
The MORIBUS system \cite{moribus} by Rohani aims to make it easy for depressed patients to remember and register their behavior in order to increase their every-day positive outcomes. The system comes with a catalogue of 54 pre-made activities such as \textit{Get ready in the morning} and \textit{Eat breakfast} and has support for patients to add their own activities by manually inputting text and a rating. When an activity is completed, the patient registers this in the app along with a short reflection. The application   is able to provide the user with statistics and summaries of activity frequency and reflections.

\subsection{MUBS}
The MUBS paper, once again by Rohani  \cite{mubs-rohani}, addresses the issue of treating depression using Behavioural Action with a mobile health solution. The MUBS application is a mobile recommender-system, which aims to help alleviate symptoms through pleasant activities. 

The application uses a catalogue of pleasant activities and give recommended activities for perform, personalized to each user. The catalogue is database of 384 common enjoyable activities where each activity has been manually labelled by researchers with an activity level, and contains a category. The recommender model uses a Naive Bayes model with a set of features representing each feature and a corresponding label representing whether it is positive or negative for that individual user. The model is designed to recommend already-known activities as well as novel activities to the user which is in the spirit of behavioural activation. It was suggested to add additional features such as \textit{location}, \textit{local weather} and \textit{ambient noise} to a future edition of the recommender system. 

\subsection{Improving the Systems}
Both the MORIBUS- and the MUBS system relies on manual input from the user in order to learn which activities to recommend in the future. This can pose a problem, especially when running longitudinal studies since participant retention may suffer if the input process is too involved for the user. Rohani is therefore a very good example of an application developer who would benefit greatly from using an automated feature generation algorithm such as the \textit{Mobility Features package}.