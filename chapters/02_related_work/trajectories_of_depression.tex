\section{Trajectories of Depression: Unobtrusive Monitoring of Depressive States by means of Smartphone Mobility Traces Analysis} 
\cite{Canzian2015}
% Main idea
This paper discusses how existing systems for diagnosing depression require the user to interact with the device. These interactions can be input such as mood-state a few times a day, which can be highly subjective and error-prone. The paper seeks to address this issue via objective patient data, called \textit{Mobility Features}, whuch were collected via unobtrusive monitoring by using smart-phone location data. There was a significant correlation between these \textit{Mobility Features} and depressive moods similar to \cite{Saeb2015}. The paper also describes concrete models to \textit{predict} changes in depressive mood by using these features. The paper uses the PHQ-8 questionnaire as reference, which is the almost the same questionnaire as the PHQ-9 \ref{appendix:questionnaires}, except for only including the first 8 questions. While this paper describes many of the same features as \cite{Saeb2015}, it does so in a much more detailed and mathematical way, such as with the \textit{Routine Index} feature, which correspond to the Circadian Rhythm feature in \cite{Saeb2015}.

\subsection{Features}
This section will describe the notation and mathematical modelling of the different entities such as places, traces and distance, described in the paper.\\

\textbf{Place: $p = (id, t_a, t_d, c)$}\\
A place is a cluster of geo-location points with a unique $id$ and a geographical center $c = (lat, lon)$. When the user arrives at the place it happens at $t_a$ and the user departs from the place at $t_d$. \\

\textbf{Mobility Trace: $MT(t_1, t_2) = [p_1, ...., p_n]$}\\
A mobility trace is a list of all places visited in the time interval $(t_1, t_2)$. $N(t_1, t_2) = n$ is the number of places visited and the time gap between $t_d (p_i)$ and $t_a (p_{i+1})$ is a period of movement.\\

\textbf{Total Distance: $D_T (t_1, t_2)$}\\
The total geodesic distance traveled in the time interval $(t_1, t_2)$.\\

\textbf{Max Distance: $D_M (t_1, t_2)$}\\
The maximum distance between any two places visited in the interval $(t_1, t_2)$.\\

\textbf{Radius of Gyration: $G(t1, t2)$}\\
The geographical area covered in the interval $(t_1, t_2)$\\

\textbf{Standard Deviation of Displacement: $\sigma_{dis}$}\\
The standard deviation between each displacement, i.e. distances from one point to another.\\

\textbf{Home Cluster: $H$}\\
The cluster most frequented cluster at 02:00, 06:00 and 20:30 on week-days.\\

\textbf{Max Distance from Home: $D_H(t_1, t_2)$}\\
The distance to the point the furthest away from home, during the interval $(t_1, t_2)$.\\

\textbf{Number of different places visited: $N_{dif} (t_1, t_2)$}\\
The number of different clusters found in the interval $(t_1, t_2)$.\\

\textbf{Number of different significant places visited: $N_{sig} (t_1, t_2)$}\\
The number of significant places visited in the interval $(t_1, t_2)$. The maximum number of significant places is 10.\\

\textbf{Routine Index: $R(t1, t2)$}\\
The degree to which the place-time distribution during the interval $(t_1, t_2)$ on a given day is similar to the distribution on other days, during the interval $(t_1, t_2)$.\\

\subsection{Smart Location Sampling}
In addition to mathematical definitions of features, the paper also provides an algorithm for sampling location in way which saves phone battery. GPS data collection is very expensive if done with a high frequency and by using cheaper sensors to turn it off when necessary can therefore save a lot of battery. Concretely the accelerometer is used by an activity recognition algorithm to determine the user's activity state. Concretely this is done by modelling the user as being in a movement state, of which there are 3: Static (S), Moving (M) and Undecided (U), where location tracking only takes place in the (U) or (M) states. \\

The definitions for the states are as follows:\\

\textbf{Static (S)}\\
The user is in the same location i.e. school/work. Never sample location data while in this state since it is unnecessary.\\

\textbf{Moving (M)}\\
The user is moving from place to place and the phone should therefore keep location tracking on, while in this state.\\

\textbf{Undecided (U)}\\
It is unknown whether the user is moving or not and moving to either state \textbf{S} or \textbf{M} is now impending. When this state is entered, the user's location $l_1$ is sampled, and after 5 minutes the location is sampled again, providing $l_2$. Then,  The distance $d = dist(l_1, l_2)$ is calculated and from this, the algorithm will transition to either \textbf{S} or \textbf{M} depending on the distance between the two locations.

The transitions are made, as follows:

\textbf{U $\rightarrow$ S}: If the distance $d < 250$ m.\\

\textbf{U $\rightarrow$ M}: If the distance $d \geq 250$ m.\\

\textbf{S $\rightarrow$ U}: If two consecutive activity recognition samples indicate user activity with over $50\%$ confidence.\\

\textbf{M $\rightarrow$  U}: If two consecutive samples are less than 250 m apart, or when an activity recognition sample indicates the user is still with more than 50\% confidence.\\

This location sampling algorithm was not used in the final product due to reasons that will be discussed in the implementation. However the algorithm  but would be ideal to build into the software package, since it ended up consuming quite a lot of battery when tracking constantly.
