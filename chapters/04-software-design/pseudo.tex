
\begin{minted}{python}
class Location(latitude, longitude):
    field latitude
    field longitude
\end{minted}

\begin{minted}{python}
class SingleLocationPoint(location, timestamp):
    field location
    field timestamp
\end{minted}


\begin{minted}{python}
class Cluster(points):
    def max_timestamp():
        return points.timestamp.max
        
    def min_timestamp():
        return points.timestamp.min
        
    def centroid():
        lat = points.location.latitude.median
        lon = points.location.longitude.median
        return Location(lat, lon)
\end{minted}

\begin{minted}{python}
class Stop(location, arrival, departure, place_id):
    field location
    field arrival
    field departure
    field place_id
\end{minted}

\begin{minted}{python}
def find_stops(D, max_dist=25):
    stops = {} # Empty set
    n = D.length
    i = 0
    while i < n:
        j = i + 1
        c = Cluster(D[i:j])
        
        # Expand cluster until the radius is exceeded
        while dist(c.centroid, D[j]) <= max_dist && (j < n):
            j += 1
            c = Cluster(D[i:j])
        
        # We have now found the first point which 
        # lies outside the cluster, and we create the stop.
        arrival = c.min_timestamp()
        departure = c.max_timestamp()
        centroid = c.centroid()
        place_id = null
        s = Stop(centroid, arrival, departure, place_id) 
        stops.append(s)
        
        # Move past already-seen data points
        i = j
    return stops
\end{minted}

\begin{minted}{python}
class Place(centroid, place_id):
    field centroid
    field place_id
\end{minted}

\begin{minted}{python}
def find_places(stops):
    places = {}
    
    # Perform DBSCAN clustering on stops
    labels = DBSCAN(stops, PLACE_DIST)
    
    # Aggregate stops on their assigned label
    groups = stops.group_by(labels)
    
    for p_id, group in (labels, groups)
        l = Location(group.lat.median, g.lon.median)
        p = Place(id, l)
        places.add(p)
        
        # Assign the current place id to each stop in the group
        for s in group:
            s.place_id = p_id
    
    return places, stops
\end{minted}
\begin{minted}{python}
class Move(stop_from, stop_to, distance):
    field stop_from
    field stop_to
    field distance
\end{minted}


% \begin{verbatim}
% CENTROID(cluster):
%     lat = cluster.latitude.median
%     lon = cluster.longitude.median
%     return (lat, lon)

% FIND_STOPS(data, max_dist):
%     stops = {} # Empty set
%     n = data.length
%     i = 0
%     WHILE i < n:
%         j = i + 1
%         # Create a cluster of points seen
%         C = D[i:j]
        
%         # Expand cluster until the radius is exceeded
%         WHILE DIST(CENTROID(C), data[j]) <= max_dist && (j < n):
%             j += 1
%             C = data[i:j]
        
%         # The point D[j] exceeds the max radius
%         arrival = C[0].timestamp # First point
%         departure = C[-1].timestamp # Last point
%         center = CENTROID(C)
%         # Create stop with unknown place-id
%         s = (center, arrival, departure, NULL) 
%         stops.append(s)
%         $a \union b$
%         # Move past clustered data points
%         i = j
%     RETURN stops
% \end{verbatim}


